% ------------------------------------------------------
% ---  Commands for typesetting sequent calculi
% ------------------------------------------------------

\newcommand {\ignore}[1]{}

% requires: proof.sty, amsmath.sty, amssymb.sty

% Contents:
%
%  Commmon macros:
%
%      linear logic connectives
%      sequent rule names: XXRuleName macros
%      common hooks: andalso, srule, etc.
%      theorem-like environments: Definition, Theorem, Lemma, ...
%      typesetting proofs: Demo, Caso, Subcaso, ...
%
%  Proof systems:
%
%      RM3 system RMTxx macros
%
%      Frames system: Fxx macros
%      Frames' system: FPxx macros
%      Frames1 system: FOnexx macros
%      Frames2 system: FTwoxx macros
%      Frames3 system: FThrxx macros
%
%      Tag-Frame system: TFxx macros
%
%      CLF systems
%
%  In general, for each proof system the following macros are defined:
%
%     XXsystem - the name of the calculus
%     XXarrow  - sequent arrow for XX
%     XXyy     - symbol yy
%     XXseq    - a sequent of the XX calculus
%     XXRseq   - a right sequent
%     XXLseq   - a left sequent

% ------------------------------------------------------
% --- linear logic connectives
% ------------------------------------------------------

% \top, \bot, and \oplus are used as such

\newcommand{\one}       {\textbf{1}}
\newcommand{\zero}      {\textbf{0}}
\newcommand{\limp}      {\mathbin{-\hspace{-0.70mm}\circ}}
\newcommand\rlimp       {\mathbin{\circ\hspace{-0.70mm}-}}
\newcommand{\iimp}      {\Rightarrow}
\newcommand{\riimp}     {\Leftarrow}
\newcommand{\bang}      {\mathop{!}}
\newcommand{\with}      {\mathbin{\&}}
\newcommand{\tensor}    {\otimes}

% ------------------------------------------------------
% --- sequent rule names
% ------------------------------------------------------

% math mode assumed in XXRuleName macros

% sequent rule name
% (#1 = proof system, #2 = connective, #3= side {left, right, none})
\newcommand {\RuleName} [3]
%    {{#2}_{#1}}
    {{#2}_{#1} \textrm{#3}}

% right rule name
% (#1 = proof system, #2 = connective)
\newcommand {\RRuleName} [2]
    {\RuleName {#1} {#2} {R}}

% left rule name
% (#1 = proof system, #2 = connective)
\newcommand {\LRuleName} [2]
    {\RuleName {#1} {#2} {L}}

% decide rule name
% (#1 = proof system, #2 = connective)
\newcommand {\DRuleName} [2]
    {\RuleName {#1} {{\textbf{decide\,}}{#2}} {}}

% pick rule name
% (#1 = proof system, #2 = connective)
\newcommand {\PRuleName} [2]
    {\RuleName {#1} {{\textbf{pick\,}}{#2}} {}}

% rewrite judgement rule name
% (#1 = proof system, #2 = connective)
\newcommand {\JRuleName} [2]
    {\RuleName {#1} {#2} {$\rewrt$}}

% axiom rule name
% (#1 = proof system, #2 = axiom name)
\newcommand {\ARuleName} [2]
    {\RuleName {#1} {\textbf{#2}} {}}

% general rule name
% (#1= proof system, #2= whatever)
\newcommand {\GRuleName}[2]
    {\RuleName {#1} {#2} {}}

% ------------------------------------------------------
% --- common hooks
% ------------------------------------------------------

% proviso hook
% (#1 = proviso)
\newcommand {\proviso}[1] {\mbox{where }#1}

% for connecting multiple premises and/or provisos
\newcommand {\andalso} {\qquad}

% rewriting symbol for the resolution judgements
\newcommand {\rewrt}  {\gg}

% ------------------------------------------------------
% --- hooks for sequent rules
% ------------------------------------------------------

% we're now using the infer macro from proofs.sty

% sequent rule
% #1= hypothesis #2= conclusion
\newcommand {\srule} [2]
   {\infer {#2} {#1}}

% named sequent rule
% #1= hypothesis #2= conclusion #3= name (XXRuleName)
\newcommand {\nsrule} [3]
   {\infer[#3] {#2} {#1}}


%% using Benjamin Pierce's bcprules.sty
%
%% sequent rule
%% #1= hypothesis #2= conclusion
%\newcommand {\srule} [2]
%   {\infrule[] {#1} {#2}}
%
%% named sequent rule
%% #1= hypothesis #2= conclusion #3= name (XXRuleName)
%\newcommand {\nsrule} [3]
%   {\infrule[#3] {#1} {#2}}


% ------------------------------------------------------
% --- theorem environments
% ------------------------------------------------------

% spanish
%\newtheorem{Teorema}{Teorema}[section]
%\newtheorem{Lema}{Lema}[section]
%\newtheorem{Proposicion}{Proposici\'{o}n}[section]
%\newtheorem{Corolario}{Corolario}[section]
%\newtheorem{Definicion}{Definici\'{o}n}[section]

% english
\newtheorem{Teorema}{Theorem}[section]
\newtheorem{Lema}{Lemma}[section]
\newtheorem{Proposicion}{Proposition}[section]
\newtheorem{Corolario}{Corollary}[section]
\newtheorem{Definicion}{Definition}[section]


% --------------------------------------
% ---  For typesetting proofs
% --------------------------------------

%% in spanish
%\newenvironment{Demo}
%      {\noindent\textbf{Demostraci\'on:\ }}
%      {\hfill $\square$ }
%
%% #1 = induction case (sequent rule, etc.)
%\newenvironment{Caso} [1]
%      {\noindent \textbf{Caso} #1}
%      {}
%
%% #1 = induction subcase (sequent rule, etc.)
%\newenvironment{Subcaso} [1]
%      {\textbf{Subcaso} #1}
%      {}

% in english
\newenvironment{Demo}
      {\textbf{Proof:\ }}
      {\hfill $\square$ }

% #1 = induction case (sequent rule, etc.)
\newenvironment{Caso} [1]
      {\noindent \textbf{Case} #1}
      {}

% #1 = induction subcase (sequent rule, etc.)
\newenvironment{Subcaso} [1]
      {\textbf{Subcase} #1}
      {}


% #1= name of assumed proof  #2= root sequent
\newcommand{\hypo}[2]
  {\deduce {#2}
           {#1}
  }

% --------------------------------------
% ---  Proof Systems
% --------------------------------------

\newcommand {\IOsep} {\backslash}

% --------------------------------------
% ---  RM3 System
% --------------------------------------

% RM3 proof system name
\newcommand {\RMTsystem} {RM_3}

% ------ RM3 sequent symbols ------

% RM3 sequent arrow symbol
\newcommand {\RMTarrow} {\Longrightarrow}

% RM3 input-output separator
\newcommand {\RMTiosep} {\backslash}

% RM syntax equality

\newcommand {\syntaxeq} {\overset{\cdot}{=}}

% ------ default RM3 sequent ------

% #1= intuitionistic #2= strict  #3= lax   #4= output       #5= top flag   #6= goal

\newcommand {\RMTseq} [6]
    {#2 ; #3 \RMTiosep  #4 {\RMTarrow}_{#5}\: #6}

\def\RMTGseq{\RMTseq{\Gamma}}

% ------ residuation judgement ------

% residuation calculus

\newcommand {\Residsystem} {\cal{R}}

% residuation rule name
% (#1 = connective)
\newcommand {\ResidRule} [1]
    {\RuleName {} {\textbf{res-}#1} {}}

% #1= clause  #2= atom #3= goal
\newcommand {\resid} [3]
    {#1 \rewrt #2 \backslash #3}

% unique labelling of a multiset
% #1= multiset
\newcommand {\unitag} [1]
    {#1^\star}

% unlabelling a uniquely labelled set
% #1= labelled set
\newcommand {\unlabel} [1]
    {[#1]}

% ------------------------------------------------------
% --- frame macros
% ------------------------------------------------------

% empty frame
\newcommand {\emptyframe} {\mathrm{nil}}

% frame constructor
\newcommand {\frcons}     {::}

% context constructor
\newcommand {\ctxcons}[2]{\mathrel{{#1}::{#2}}}

\newcommand{\emptyctx}{\mathrm{nil}}

% insert a element in a frame
\newcommand {\frins}      {\triangleleft}

% to flatten a frame
\newcommand {\frflat} [1] {\widehat{#1}}

% inclusion of frames
\newcommand {\frleq}      {\sqsubseteq}

% union of frames
\newcommand {\frplus}     {\sqcup}

% frame difference
\newcommand {\frdiff}     {-}

% flattened frame difference
\newcommand {\frflatdif}     {\frflat{\frdiff}}

% --------------------------------------
% ---  Frames System
% --------------------------------------

% Frames proof system name
\newcommand {\Fsystem} {\cal{F}}

% Frames proof system symbols

\newcommand {\Farrow} {\Longrightarrow}
\newcommand {\Fiosep} {\slash}

% ------ F right sequent ------

% #1= intuitionistic input #2= strict input  #3= optional input  #4= goal  #5= output  #6= top flag
\newcommand  {\FRseq}  [6] {#1; #2; #3 \Farrow #4 \Fiosep #5; #6}

% ------ F left sequent ------

% #1= intuitionistic input #2= strict input  #3= optional input  #4= clause (focus) #5= atom  #6= output  #7= top flag
\newcommand  {\FLseq}  [7] {#1; #2; #3 \Farrow #4 \rewrt #5 \Fiosep #6; #7}

% --------------------------------------
% ---  Frames' System
% --------------------------------------

% Frames' proof system name
\newcommand {\Fpsystem} {\cal{F}'}

% ------ default F' sequent ------

% #1= strict input  #2= optional input  #3= goal  #4= output  #5= top flag
\newcommand  {\Fpseq}  [5] {#1; #2 \Farrow #3 \Fiosep #4; #5}

% ------------------------------------
% --- First refinement of Frames
% ------------------------------------

% F1 proof system name
\newcommand {\FOnesystem} {{\Fsystem}_{1}}

% F1 right sequent
% #1= strict input  #2= optional input  #3= goal  #4= output  #5= top flag
\newcommand {\FOneRseq} [5] {#1; #2 \Farrow #3 \Fiosep #4; #5}

% F1 left sequent
% #1= strict input  #2= optional input  #3= clause (focus) #4= atom  #5= output  #6= top flag
\newcommand {\FOneLseq} [6] {#1; #2 \Farrow #3 \rewrt #4 \Fiosep #5; #6}

% ------------------------------------
% --- Second refinement of Frames
% ------------------------------------

% F2 proof system name
\newcommand {\FTwosystem} {{\Fsystem}_{2}}

% F2 right sequent
% #1= input program  #2= goal  #3= output #4= top flag
\newcommand {\FTwoRseq} [4] {#1 \Farrow #2 \Fiosep #3; #4}

% F2 left sequent
% #1= input program  #2= clause #3= atom  #4= output #5= top flag
\newcommand {\FTwoLseq} [5] {#1 \Farrow #2 \rewrt #3 \Fiosep #4; #5}

% ------------------------------------
% --- Third refinement of Frames
% ------------------------------------

% F3 proof system name
\newcommand {\FThrsystem} {{\Fsystem}_{3}}

% F3 right sequent
% #1= input program  #2= goal  #3= output #4= top flag
\newcommand {\FThrRseq} [4] {#1 \Farrow #2 \Fiosep #3; #4}

% F3 left sequent
% #1= input program  #2= clause #3= atom  #4= output #5= top flag
\newcommand {\FThrLseq} [5] {#1 \Farrow #2 \rewrt #3 \Fiosep #4; #5}

% --------------------------------------
% ---  Tag-Frame System
% --------------------------------------

% TF proof system name
\newcommand {\TFsystem} {\cal{TF}}

% TF' proof system name
\newcommand {\TFpsystem} {\cal{TF}'}

% ------ TF sequent symbols ------

% TF sequent arrow symbol
\newcommand {\TFarrow} {\longrightarrow}
\renewcommand {\TFarrow} {\xrightarrow}

% TF input-output separator
\newcommand {\TFiosep} {\diagdown}

% TF tag standard signature
\newcommand {\TFsig} {\Sigma}

% universal set of tags
\newcommand {\Tags} {\cal{T}}

% disjoint union
\newcommand {\udot} {\dot\cup}

% selection of part of a context affected by given tags
% #1 = sequence of tagged formulas  #2= set of tags
\newcommand {\tagsel} [2]
    {\left[#1\right]_{#2}}

\newcommand {\tagself} [3]
    {\left[#1\right]_{#2}^{#3}}

\newcommand {\tagssel} [2]
    {\left\llbracket{#1}\right\rrbracket_{#2}}

%multiset
\newcommand{\multi}[1]{\{ {#1} \}}
\def\mand{\mbox{ and }}
\def\mor{\mbox{ or }}

% tags appearing in a context
\newcommand {\tagsof} [1]
    {\left\langle #1 \right\rangle}

% ------ default TF sequent ------

% #1 = signature #2= input  #3= output   #4= frame       #5= consumption markers   #8= goal        INPUT
%                         #6= ext. consumption markers  #7= top flag                               OUTPUT
%\newcommand {\TFseq} [8]
%    {#1: #2 \TFiosep  #3 \overset{#4 \quad #5}{\underset{#6 \quad #7} {\TFarrow}}  #8}
\newcommand {\TFseq} [8]
    {#1: #2 \TFiosep  #3 \TFarrow[#6 \quad #7]{#4 \quad #5}  #8}


% TF sequent no signature (drops #1)
% #1 = signature #2= input  #3= output   #4= frame       #5= consumption markers   #8= goal        INPUT
%                         #6= ext. consumption markers  #7= top flag                               OUTPUT
%\newcommand {\TFseqnosig} [8]
%    {#2 \TFiosep  #3 \overset{#4 \quad #5}{\underset{#6 \quad #7} {\TFarrow}}  #8}
\newcommand {\TFseqnosig} [8]
    {#2 \TFiosep  #3 \TFarrow[#6 \quad #7]{#4 \quad #5} #8}

% --------------------------------------
% ---  Tag-Frame-Phi System
% --------------------------------------

% this is Tag-Frame modified to explicitly compute \phi; 
% that is, the set of tags exported by \with rules

% TFH' proof system name
\newcommand {\TFHpsystem} {\mathcal{TF}'_\phi}

% ------ default TFH sequent ------

% #1 = signature #2= input  #3= output   #4= frame       #5= consumption markers   #9= goal        INPUT
%              #6= ext. consumption markers  #7= tags exported by &  #8= top flag                  OUTPUT
%\newcommand {\TFHseq} [9]
%    {#1: #2 \TFiosep  #3 \overset{#4 \quad #5}{\underset{#6 \quad #7 \quad #8} {\TFarrow}}  #9}
\newcommand {\TFHseq} [9]
    {#1: #2 \TFiosep  #3 \TFarrow[#6 \quad #7 \quad #8] {#4 \quad #5}  #9}

% --------------------------------------
% ---  Tag-Frame-Beta System
% --------------------------------------

% TFH' proof system name
\newcommand {\TFBpsystem} {\mathcal{TF}'_\beta}

% ------ default TFB sequent ------

% #1 = signature #2= input  #3= output   #4= frame       #5= consumption markers   #8= goal        INPUT
%                             #6= tags exported by &  #7= top flag                                 OUTPUT
%\newcommand {\TFBseq} [8]
%    {#1: #2 \TFiosep  #3 \overset{#4 \quad #5}{\underset{#6 \quad #7} {\TFarrow}}  #8}

\newcommand {\TFBseq} [8]
    {#1: #2 \TFiosep  #3 \TFarrow[#6 \quad #7]{#4 \quad #5}  #8}

\newcommand {\TFBGseq} [8]
    {#1:\: #2 \TFiosep  #3 \TFarrow[#6 \quad #7]{#4 \quad #5}  #8}

\newcommand {\TFBseqG} [9]
    {#1:\: #2 ; #3 \TFiosep  #4 \TFarrow[#5 \quad #6]{#7 \quad #8}  #9}

% --------------------------------------
% ---  Tag-Frame-Plus System
% --------------------------------------

\newcommand {\TFPpsystem} {\mathcal{TF}+}

\newcommand {\TFPseqq} [6]
    {#1 \TFiosep  #2 {\TFarrow{#3 \quad #4}}_{#5}\:  #6}

\newcommand{\TFPseq}[5]{#2 \TFiosep #4 \:;\: \TFPseqq}

\newcommand{\TFPGseq}[4]{#1 \TFiosep #3 \:;\: \TFPseqq}

\newcommand{\lookup}[5]{#1\TFiosep #2\models #4 \leadsto #5}
\newcommand{\card}[1]{\bm{\#}(#1)}
\newcommand{\val}[1]{#1}
\newcommand{\alias}[1]{#1}

\newcommand{\aliasof}[2]{\texttt{alias\_of}(#1,#2)}

% --------------------------------------
% ---  Tag-Frame-Fast System
% --------------------------------------

\newcommand {\TFapsystem} {\mathcal{TF}'_\beta}

\newcommand {\TFPapsystem} {\mathcal{TF}\lightning}
\newcommand {\TFPaseqq} [6]
    {#1 \TFiosep  #2 {\TFarrow{#3 \quad #4}}_{#5}\: #6}

\newcommand{\TFPaGseq}[4]{\Lambda ; #1 \TFiosep #3 \:;\: \TFPaseqq}
\newcommand{\TFPaLseq}[5]{#1 ; #2 \TFiosep #4 \:;\: \TFPaseqq}

\newcommand{\fst}[1]{\mathrm{fst}(#1)}
\newcommand{\snd}[1]{\mathrm{snd}(#1)}

\newcommand{\EWsystem}{\mathcal{TF}\merge}

% ------ default TFa sequent ------

% #1 = signature #2= input  #3= output   #4= frame       #5= consumption markers   #8= goal        INPUT
%                             #6= tags exported by &  #7= top flag                                 OUTPUT
%\newcommand {\TFaseq} [8]
%    {#1: #2 \TFiosep  #3 \overset{#4 \quad #5}{\underset{#6 \quad #7} {\TFarrow}}  #8}

\newcommand {\TFaseq} [7]
    {#2 \TFiosep  #3 {\TFarrow{#4 \quad #5}}_{#6}\:  #7}

\newcommand {\TFaGseq} [6]
    {#1 \TFiosep  #2 {\TFarrow{#3 \quad #4}}_{#5}\:  #6}

\newcommand{\maptag}[3]{\mathrel{\merge(#1,#2,#3)}}


\newcommand {\TFFpsystem} {\mathcal{TF}\lightning}


% --------------------------------------
% --- CLF
% --------------------------------------

% a sequent calculus presentation of CLF
% Note: proof-terms dropped

\newcommand {\CLFsystem} {CLF}

%--- expression checking (EC)

\newcommand{\CLFECarrow}{\xleftarrow}

% #1= intuitionistic context  #2= linear context  #3= sync type
\newcommand{\CLFECseq}[3]
      {#1 ; #2 \CLFECarrow{} #3}

%--- normal object cheking (NC)

\newcommand{\CLFNCarrow}{\xLeftarrow}

% #1= intuitionistic context  #2= linear context  #3= type
\newcommand{\CLFNCseq}[3]
      {#1 ; #2 \CLFNCarrow{} #3}

%--- monadic object cheking (MC)

\newcommand{\CLFMCarrow}{\xLeftarrow}

% #1= intuitionistic context  #2= linear context  #3= sync type
\newcommand{\CLFMCseq}[3]
      {#1 ; #2 \CLFMCarrow{} #3}

%--- pattern expansion (PE)

\newcommand{\CLFPEarrow}{\xleftarrow}

% #1= intuitionistic context  #2= linear context  #3= patterns  #4= async type
\newcommand{\CLFPEseq}[4]
      {#1 ; #2 \, \vert \, #3 \CLFPEarrow{} #4}

%--- atomic object inference (AI)

\newcommand{\CLFAIarrow}{\xRightarrow}

% #1= intuitionistic context  #2= linear context  #3= async type
\newcommand{\CLFAIseq}[3]
      {#1 ; #2 \CLFAIarrow{} #3}

%--- forward chaining

\newcommand{\CLFFCarrow}{\Downarrow}

\newcommand{\commit}{\mathbf{commit}}

% #1= intuitionistic context  #2= linear context  #3= async type #4= monadic type
\newcommand{\CLFFCseq}[4]
      {#1 ; #2 \, \vert \, #3 \CLFFCarrow  #4}


% --------------------------------------
% --- CLF IO_\top
% --------------------------------------

% an IO_\top presentation of CLF
% Note: proof-terms dropped

\newcommand {\CLFIOsystem} {CLF_{IO_{\top}}}

%--- expression checking (EC)

% #1= intuitionistic context #2= input #3= output #4= top flag  #5= sync type
\newcommand{\CLFIOECseq}[5]
      {#1 ; #2 \IOsep #3 \CLFECarrow[#4]{} #5}

%--- normal object cheking (NC)

% #1= intuitionistic context #2= input #3= output #4= top flag  #5= type
\newcommand{\CLFIONCseq}[5]
      {#1 ; #2 \IOsep #3 \CLFNCarrow[#4]{} #5}

%--- monadic object cheking (MC)

% #1= intuitionistic context #2= input #3= output #4= top flag  #5= sync type
\newcommand{\CLFIOMCseq}[5]
      {#1 ; #2 \IOsep #3 \CLFMCarrow[#4]{} #5}

%--- pattern expansion (PE)

% #1= intuitionistic context #2= input #3= output  #4= patterns #5= top flag #6= async type
\newcommand{\CLFIOPEseq}[6]
      {#1 ; #2 \IOsep #3 \, \vert \, #4 \CLFPEarrow[#5]{} #6}

%--- atomic object inference (AI)

% #1= intuitionistic context #2= input #3= output #4= top flag  #5= async type
\newcommand{\CLFIOAIseq}[5]
      {#1 ; #2 \IOsep #3 \CLFAIarrow[#4]{} #5}

%--- forward chaining

% #1= intuitionistic context  #2= input #3= output  #4= async type #5= top flag   #6= monadic type
\newcommand{\CLFIOFCseq}[6]
      {#1 ; #2 \IOsep #3 \, \vert \, #4 \CLFFCarrow_{#5}  #6}


% --------------------------------------
% --- CLF Frames
% --------------------------------------

% a Frames presentation of CLF
% Note: proof-terms dropped

\newcommand {\CLFFRsystem} {CLF_{\mathcal{F}}}

%--- expression checking (EC)

% #1= intuitionistic context #2= strict #3= input #4= output #5= top flag  #6= sync type
\newcommand{\CLFFRECseq}[6]
      {#1 ; #2 ; #3 \IOsep #4 \CLFECarrow[#5]{} #6}

%--- normal object cheking (NC)

% #1= intuitionistic context #2= strict #3= input #4= output #5= top flag  #6= type
\newcommand{\CLFFRNCseq}[6]
      {#1 ; #2 ; #3 \IOsep #4 \CLFNCarrow[#5]{} #6}

%--- monadic object cheking (MC)

% #1= intuitionistic context #2= stritc #3= input #4= output #5= top flag  #6= sync type
\newcommand{\CLFFRMCseq}[6]
      {#1 ; #2 ; #3 \IOsep #4 \CLFMCarrow[#5]{} #6}

%--- atomic object inference (AI)

% #1= intuitionistic context #2= strict #3= input #4= output #5= top flag  #6= async type
\newcommand{\CLFFRAIseq}[6]
      {#1 ; #2 ; #3 \IOsep #4 \CLFAIarrow[#5]{} #6}

%--- pattern expansion (PE)

% #1= intuitionistic context #2= strict #3= input #4= output  #5= patterns #6= top flag #7= async type
\newcommand{\CLFFRPEseq}[7]
      {#1 ; #2 ; #3 \IOsep #4 \, \vert \, #5 \CLFPEarrow[#6]{} #7}


%--- forward chaining

% #1= intuitionistic context  #2= strict #3= input #4= output  #5= async type #6= top flag   #7= monadic type
\newcommand{\CLFFRFCseq}[7]
      {#1 ; #2 ; #3 \IOsep #4 \, \vert \, #5 \CLFFCarrow_{#6}  #7}


% --------------------------------------
% ---  CLF Tag-Frame-Fast
% --------------------------------------

% a Tag-Frame-Fast presentation of CLF
% Note: proof-terms and intuitionistic context dropped

\newcommand {\CLFTFsystem} {CLF_{TF}}

%--- expression checking (EC)

%  #1=Stack id  #2=Alias in   #3=Alias out   #4=Input  #5=Output
%  #6= strict tag   #7=delete tag  #8=top flag #9= goal
\newcommand {\CLFTFECseq}[9]
    {#1; \; #2 \TFiosep #3; \; #4 \TFiosep #5  {\CLFECarrow[#8]{#6 \quad #7}}  #9}

%--- normal object cheking (NC)

%  #1=Stack id  #2=Alias in   #3=Alias out   #4=Input  #5=Output
%  #6= strict tag   #7=delete tag  #8=top flag #9= goal
\newcommand {\CLFTFNCseq}[9]
    {#1; \; #2 \TFiosep #3; \; #4 \TFiosep #5  {\CLFNCarrow[#8]{#6 \quad #7}} #9}

%--- monadic object cheking (MC)

%  #1=Stack id  #2=Alias in   #3=Alias out   #4=Input  #5=Output
%  #6= strict tag   #7=delete tag  #8=top flag #9= goal
\newcommand {\CLFTFMCseq}[9]
    {#1; \; #2 \TFiosep #3; \; #4 \TFiosep #5  {\CLFMCarrow[#8]{#6 \quad #7}} #9}

%--- pattern expansion (PE)

% 10 arguments, must split

% this is to write the rest of parameters
\newcommand {\CLFTFPErest} [1] {#1}

%  #1=Stack id  #2=Alias in   #3=Alias out   #4=Input  #5=Output  #6= Patterns
%  #7= strict tag   #8=delete tag  #9=top flag #10= goal
\newcommand {\CLFTFPEseq}[9]
    {#1; \; #2 \TFiosep #3; \; #4 \TFiosep #5 \; \vert #6  {\CLFPEarrow[#9]{#7 \quad #8}} \CLFTFPErest}


% --------------------------------------
% ---  Extra stuff (mostly Jeff's)
% --------------------------------------


\def\emptypf{\emptyset}
\def\bnfas{\mathrel{::=}}
\def\bnfalt{\mid}
\def\S{\sigma}
\def\St{\Sigma}
\def\D{\Delta}
\def\L{\alpha}
\def\tt{\mathrm{t}\!\mathrm{t}}
\def\ff{\mathrm{f}\!\mathrm{f}}

\def\none{\blacklozenge}

\newcommand{\dom}[1]{\mathrm{dom}(#1)}
\newcommand{\cod}[1]{\mathrm{cod}(#1)}

\newcommand{\hs}[1]{\hspace*{#1em}}

\newcommand {\rsq}[4]{#2 \gg #3 \TFiosep #4}


\def\above#1#2{\begin{array}[b]{c}\relax #1\\ \relax #2\end{array}}
\def\derI{\mathcal{D}_1}
\def\derII{\mathcal{D}_2}

\newcommand{\IOTseq}[4]{#1\TFiosep #2\:\Longrightarrow_{#3}\: #4}
\newcommand{\Fseq}[5]{#1;#2\TFiosep #3 \longrightarrow_{#4} \: #5 }


%%% Local Variables: 
%%% mode: plain-tex
%%% TeX-master: "E:\\lolli_clf\\opsem"
%%% End: 
