\documentclass{sig-alt}

\usepackage{proof}
%\usepackage{palatino}
\usepackage{array}
\usepackage{stmaryrd}
\usepackage{bm}

% ------------------------------------------------------
% ---  Commands for typesetting sequent calculi
% ------------------------------------------------------

\newcommand {\ignore}[1]{}

% requires: proof.sty, amsmath.sty, amssymb.sty

% Contents:
%
%  Commmon macros:
%
%      linear logic connectives
%      sequent rule names: XXRuleName macros
%      common hooks: andalso, srule, etc.
%      theorem-like environments: Definition, Theorem, Lemma, ...
%      typesetting proofs: Demo, Caso, Subcaso, ...
%
%  Proof systems:
%
%      RM3 system RMTxx macros
%
%      Frames system: Fxx macros
%      Frames' system: FPxx macros
%      Frames1 system: FOnexx macros
%      Frames2 system: FTwoxx macros
%      Frames3 system: FThrxx macros
%
%      Tag-Frame system: TFxx macros
%
%      CLF systems
%
%  In general, for each proof system the following macros are defined:
%
%     XXsystem - the name of the calculus
%     XXarrow  - sequent arrow for XX
%     XXyy     - symbol yy
%     XXseq    - a sequent of the XX calculus
%     XXRseq   - a right sequent
%     XXLseq   - a left sequent

% ------------------------------------------------------
% --- linear logic connectives
% ------------------------------------------------------

% \top, \bot, and \oplus are used as such

\newcommand{\one}       {\textbf{1}}
\newcommand{\zero}      {\textbf{0}}
\newcommand{\limp}      {\mathbin{-\hspace{-0.70mm}\circ}}
\newcommand\rlimp       {\mathbin{\circ\hspace{-0.70mm}-}}
\newcommand{\iimp}      {\Rightarrow}
\newcommand{\riimp}     {\Leftarrow}
\newcommand{\bang}      {\mathop{!}}
\newcommand{\with}      {\mathbin{\&}}
\newcommand{\tensor}    {\otimes}

% ------------------------------------------------------
% --- sequent rule names
% ------------------------------------------------------

% math mode assumed in XXRuleName macros

% sequent rule name
% (#1 = proof system, #2 = connective, #3= side {left, right, none})
\newcommand {\RuleName} [3]
%    {{#2}_{#1}}
    {{#2}_{#1} \textrm{#3}}

% right rule name
% (#1 = proof system, #2 = connective)
\newcommand {\RRuleName} [2]
    {\RuleName {#1} {#2} {R}}

% left rule name
% (#1 = proof system, #2 = connective)
\newcommand {\LRuleName} [2]
    {\RuleName {#1} {#2} {L}}

% decide rule name
% (#1 = proof system, #2 = connective)
\newcommand {\DRuleName} [2]
    {\RuleName {#1} {{\textbf{decide\,}}{#2}} {}}

% pick rule name
% (#1 = proof system, #2 = connective)
\newcommand {\PRuleName} [2]
    {\RuleName {#1} {{\textbf{pick\,}}{#2}} {}}

% rewrite judgement rule name
% (#1 = proof system, #2 = connective)
\newcommand {\JRuleName} [2]
    {\RuleName {#1} {#2} {$\rewrt$}}

% axiom rule name
% (#1 = proof system, #2 = axiom name)
\newcommand {\ARuleName} [2]
    {\RuleName {#1} {\textbf{#2}} {}}

% general rule name
% (#1= proof system, #2= whatever)
\newcommand {\GRuleName}[2]
    {\RuleName {#1} {#2} {}}

% ------------------------------------------------------
% --- common hooks
% ------------------------------------------------------

% proviso hook
% (#1 = proviso)
\newcommand {\proviso}[1] {\mbox{where }#1}

% for connecting multiple premises and/or provisos
\newcommand {\andalso} {\qquad}

% rewriting symbol for the resolution judgements
\newcommand {\rewrt}  {\gg}

% ------------------------------------------------------
% --- hooks for sequent rules
% ------------------------------------------------------

% we're now using the infer macro from proofs.sty

% sequent rule
% #1= hypothesis #2= conclusion
\newcommand {\srule} [2]
   {\infer {#2} {#1}}

% named sequent rule
% #1= hypothesis #2= conclusion #3= name (XXRuleName)
\newcommand {\nsrule} [3]
   {\infer[#3] {#2} {#1}}


%% using Benjamin Pierce's bcprules.sty
%
%% sequent rule
%% #1= hypothesis #2= conclusion
%\newcommand {\srule} [2]
%   {\infrule[] {#1} {#2}}
%
%% named sequent rule
%% #1= hypothesis #2= conclusion #3= name (XXRuleName)
%\newcommand {\nsrule} [3]
%   {\infrule[#3] {#1} {#2}}


% ------------------------------------------------------
% --- theorem environments
% ------------------------------------------------------

% spanish
%\newtheorem{Teorema}{Teorema}[section]
%\newtheorem{Lema}{Lema}[section]
%\newtheorem{Proposicion}{Proposici\'{o}n}[section]
%\newtheorem{Corolario}{Corolario}[section]
%\newtheorem{Definicion}{Definici\'{o}n}[section]

% english
\newtheorem{Teorema}{Theorem}[section]
\newtheorem{Lema}{Lemma}[section]
\newtheorem{Proposicion}{Proposition}[section]
\newtheorem{Corolario}{Corollary}[section]
\newtheorem{Definicion}{Definition}[section]


% --------------------------------------
% ---  For typesetting proofs
% --------------------------------------

%% in spanish
%\newenvironment{Demo}
%      {\noindent\textbf{Demostraci\'on:\ }}
%      {\hfill $\square$ }
%
%% #1 = induction case (sequent rule, etc.)
%\newenvironment{Caso} [1]
%      {\noindent \textbf{Caso} #1}
%      {}
%
%% #1 = induction subcase (sequent rule, etc.)
%\newenvironment{Subcaso} [1]
%      {\textbf{Subcaso} #1}
%      {}

% in english
\newenvironment{Demo}
      {\textbf{Proof:\ }}
      {\hfill $\square$ }

% #1 = induction case (sequent rule, etc.)
\newenvironment{Caso} [1]
      {\noindent \textbf{Case} #1}
      {}

% #1 = induction subcase (sequent rule, etc.)
\newenvironment{Subcaso} [1]
      {\textbf{Subcase} #1}
      {}


% #1= name of assumed proof  #2= root sequent
\newcommand{\hypo}[2]
  {\deduce {#2}
           {#1}
  }

% --------------------------------------
% ---  Proof Systems
% --------------------------------------

\newcommand {\IOsep} {\backslash}

% --------------------------------------
% ---  RM3 System
% --------------------------------------

% RM3 proof system name
\newcommand {\RMTsystem} {RM_3}

% ------ RM3 sequent symbols ------

% RM3 sequent arrow symbol
\newcommand {\RMTarrow} {\Longrightarrow}

% RM3 input-output separator
\newcommand {\RMTiosep} {\backslash}

% RM syntax equality

\newcommand {\syntaxeq} {\overset{\cdot}{=}}

% ------ default RM3 sequent ------

% #1= intuitionistic #2= strict  #3= lax   #4= output       #5= top flag   #6= goal

\newcommand {\RMTseq} [6]
    {#2 ; #3 \RMTiosep  #4 {\RMTarrow}_{#5}\: #6}

\def\RMTGseq{\RMTseq{\Gamma}}

% ------ residuation judgement ------

% residuation calculus

\newcommand {\Residsystem} {\cal{R}}

% residuation rule name
% (#1 = connective)
\newcommand {\ResidRule} [1]
    {\RuleName {} {\textbf{res-}#1} {}}

% #1= clause  #2= atom #3= goal
\newcommand {\resid} [3]
    {#1 \rewrt #2 \backslash #3}

% unique labelling of a multiset
% #1= multiset
\newcommand {\unitag} [1]
    {#1^\star}

% unlabelling a uniquely labelled set
% #1= labelled set
\newcommand {\unlabel} [1]
    {[#1]}

% ------------------------------------------------------
% --- frame macros
% ------------------------------------------------------

% empty frame
\newcommand {\emptyframe} {\mathrm{nil}}

% frame constructor
\newcommand {\frcons}     {::}

% context constructor
\newcommand {\ctxcons}[2]{\mathrel{{#1}::{#2}}}

\newcommand{\emptyctx}{\mathrm{nil}}

% insert a element in a frame
\newcommand {\frins}      {\triangleleft}

% to flatten a frame
\newcommand {\frflat} [1] {\widehat{#1}}

% inclusion of frames
\newcommand {\frleq}      {\sqsubseteq}

% union of frames
\newcommand {\frplus}     {\sqcup}

% frame difference
\newcommand {\frdiff}     {-}

% flattened frame difference
\newcommand {\frflatdif}     {\frflat{\frdiff}}

% --------------------------------------
% ---  Frames System
% --------------------------------------

% Frames proof system name
\newcommand {\Fsystem} {\cal{F}}

% Frames proof system symbols

\newcommand {\Farrow} {\Longrightarrow}
\newcommand {\Fiosep} {\slash}

% ------ F right sequent ------

% #1= intuitionistic input #2= strict input  #3= optional input  #4= goal  #5= output  #6= top flag
\newcommand  {\FRseq}  [6] {#1; #2; #3 \Farrow #4 \Fiosep #5; #6}

% ------ F left sequent ------

% #1= intuitionistic input #2= strict input  #3= optional input  #4= clause (focus) #5= atom  #6= output  #7= top flag
\newcommand  {\FLseq}  [7] {#1; #2; #3 \Farrow #4 \rewrt #5 \Fiosep #6; #7}

% --------------------------------------
% ---  Frames' System
% --------------------------------------

% Frames' proof system name
\newcommand {\Fpsystem} {\cal{F}'}

% ------ default F' sequent ------

% #1= strict input  #2= optional input  #3= goal  #4= output  #5= top flag
\newcommand  {\Fpseq}  [5] {#1; #2 \Farrow #3 \Fiosep #4; #5}

% ------------------------------------
% --- First refinement of Frames
% ------------------------------------

% F1 proof system name
\newcommand {\FOnesystem} {{\Fsystem}_{1}}

% F1 right sequent
% #1= strict input  #2= optional input  #3= goal  #4= output  #5= top flag
\newcommand {\FOneRseq} [5] {#1; #2 \Farrow #3 \Fiosep #4; #5}

% F1 left sequent
% #1= strict input  #2= optional input  #3= clause (focus) #4= atom  #5= output  #6= top flag
\newcommand {\FOneLseq} [6] {#1; #2 \Farrow #3 \rewrt #4 \Fiosep #5; #6}

% ------------------------------------
% --- Second refinement of Frames
% ------------------------------------

% F2 proof system name
\newcommand {\FTwosystem} {{\Fsystem}_{2}}

% F2 right sequent
% #1= input program  #2= goal  #3= output #4= top flag
\newcommand {\FTwoRseq} [4] {#1 \Farrow #2 \Fiosep #3; #4}

% F2 left sequent
% #1= input program  #2= clause #3= atom  #4= output #5= top flag
\newcommand {\FTwoLseq} [5] {#1 \Farrow #2 \rewrt #3 \Fiosep #4; #5}

% ------------------------------------
% --- Third refinement of Frames
% ------------------------------------

% F3 proof system name
\newcommand {\FThrsystem} {{\Fsystem}_{3}}

% F3 right sequent
% #1= input program  #2= goal  #3= output #4= top flag
\newcommand {\FThrRseq} [4] {#1 \Farrow #2 \Fiosep #3; #4}

% F3 left sequent
% #1= input program  #2= clause #3= atom  #4= output #5= top flag
\newcommand {\FThrLseq} [5] {#1 \Farrow #2 \rewrt #3 \Fiosep #4; #5}

% --------------------------------------
% ---  Tag-Frame System
% --------------------------------------

% TF proof system name
\newcommand {\TFsystem} {\cal{TF}}

% TF' proof system name
\newcommand {\TFpsystem} {\cal{TF}'}

% ------ TF sequent symbols ------

% TF sequent arrow symbol
\newcommand {\TFarrow} {\longrightarrow}
\renewcommand {\TFarrow} {\xrightarrow}

% TF input-output separator
\newcommand {\TFiosep} {\diagdown}

% TF tag standard signature
\newcommand {\TFsig} {\Sigma}

% universal set of tags
\newcommand {\Tags} {\cal{T}}

% disjoint union
\newcommand {\udot} {\dot\cup}

% selection of part of a context affected by given tags
% #1 = sequence of tagged formulas  #2= set of tags
\newcommand {\tagsel} [2]
    {\left[#1\right]_{#2}}

\newcommand {\tagself} [3]
    {\left[#1\right]_{#2}^{#3}}

\newcommand {\tagssel} [2]
    {\left\llbracket{#1}\right\rrbracket_{#2}}

%multiset
\newcommand{\multi}[1]{\{ {#1} \}}
\def\mand{\mbox{ and }}
\def\mor{\mbox{ or }}

% tags appearing in a context
\newcommand {\tagsof} [1]
    {\left\langle #1 \right\rangle}

% ------ default TF sequent ------

% #1 = signature #2= input  #3= output   #4= frame       #5= consumption markers   #8= goal        INPUT
%                         #6= ext. consumption markers  #7= top flag                               OUTPUT
%\newcommand {\TFseq} [8]
%    {#1: #2 \TFiosep  #3 \overset{#4 \quad #5}{\underset{#6 \quad #7} {\TFarrow}}  #8}
\newcommand {\TFseq} [8]
    {#1: #2 \TFiosep  #3 \TFarrow[#6 \quad #7]{#4 \quad #5}  #8}


% TF sequent no signature (drops #1)
% #1 = signature #2= input  #3= output   #4= frame       #5= consumption markers   #8= goal        INPUT
%                         #6= ext. consumption markers  #7= top flag                               OUTPUT
%\newcommand {\TFseqnosig} [8]
%    {#2 \TFiosep  #3 \overset{#4 \quad #5}{\underset{#6 \quad #7} {\TFarrow}}  #8}
\newcommand {\TFseqnosig} [8]
    {#2 \TFiosep  #3 \TFarrow[#6 \quad #7]{#4 \quad #5} #8}

% --------------------------------------
% ---  Tag-Frame-Phi System
% --------------------------------------

% this is Tag-Frame modified to explicitly compute \phi; 
% that is, the set of tags exported by \with rules

% TFH' proof system name
\newcommand {\TFHpsystem} {\mathcal{TF}'_\phi}

% ------ default TFH sequent ------

% #1 = signature #2= input  #3= output   #4= frame       #5= consumption markers   #9= goal        INPUT
%              #6= ext. consumption markers  #7= tags exported by &  #8= top flag                  OUTPUT
%\newcommand {\TFHseq} [9]
%    {#1: #2 \TFiosep  #3 \overset{#4 \quad #5}{\underset{#6 \quad #7 \quad #8} {\TFarrow}}  #9}
\newcommand {\TFHseq} [9]
    {#1: #2 \TFiosep  #3 \TFarrow[#6 \quad #7 \quad #8] {#4 \quad #5}  #9}

% --------------------------------------
% ---  Tag-Frame-Beta System
% --------------------------------------

% TFH' proof system name
\newcommand {\TFBpsystem} {\mathcal{TF}'_\beta}

% ------ default TFB sequent ------

% #1 = signature #2= input  #3= output   #4= frame       #5= consumption markers   #8= goal        INPUT
%                             #6= tags exported by &  #7= top flag                                 OUTPUT
%\newcommand {\TFBseq} [8]
%    {#1: #2 \TFiosep  #3 \overset{#4 \quad #5}{\underset{#6 \quad #7} {\TFarrow}}  #8}

\newcommand {\TFBseq} [8]
    {#1: #2 \TFiosep  #3 \TFarrow[#6 \quad #7]{#4 \quad #5}  #8}

\newcommand {\TFBGseq} [8]
    {#1:\: #2 \TFiosep  #3 \TFarrow[#6 \quad #7]{#4 \quad #5}  #8}

\newcommand {\TFBseqG} [9]
    {#1:\: #2 ; #3 \TFiosep  #4 \TFarrow[#5 \quad #6]{#7 \quad #8}  #9}

% --------------------------------------
% ---  Tag-Frame-Plus System
% --------------------------------------

\newcommand {\TFPpsystem} {\mathcal{TF}+}

\newcommand {\TFPseqq} [6]
    {#1 \TFiosep  #2 {\TFarrow{#3 \quad #4}}_{#5}\:  #6}

\newcommand{\TFPseq}[5]{#2 \TFiosep #4 \:;\: \TFPseqq}

\newcommand{\TFPGseq}[4]{#1 \TFiosep #3 \:;\: \TFPseqq}

\newcommand{\lookup}[5]{#1\TFiosep #2\models #4 \leadsto #5}
\newcommand{\card}[1]{\bm{\#}(#1)}
\newcommand{\val}[1]{#1}
\newcommand{\alias}[1]{#1}

\newcommand{\aliasof}[2]{\texttt{alias\_of}(#1,#2)}

% --------------------------------------
% ---  Tag-Frame-Fast System
% --------------------------------------

\newcommand {\TFapsystem} {\mathcal{TF}'_\beta}

\newcommand {\TFPapsystem} {\mathcal{TF}\lightning}
\newcommand {\TFPaseqq} [6]
    {#1 \TFiosep  #2 {\TFarrow{#3 \quad #4}}_{#5}\: #6}

\newcommand{\TFPaGseq}[4]{\Lambda ; #1 \TFiosep #3 \:;\: \TFPaseqq}
\newcommand{\TFPaLseq}[5]{#1 ; #2 \TFiosep #4 \:;\: \TFPaseqq}

\newcommand{\fst}[1]{\mathrm{fst}(#1)}
\newcommand{\snd}[1]{\mathrm{snd}(#1)}

\newcommand{\EWsystem}{\mathcal{TF}\merge}

% ------ default TFa sequent ------

% #1 = signature #2= input  #3= output   #4= frame       #5= consumption markers   #8= goal        INPUT
%                             #6= tags exported by &  #7= top flag                                 OUTPUT
%\newcommand {\TFaseq} [8]
%    {#1: #2 \TFiosep  #3 \overset{#4 \quad #5}{\underset{#6 \quad #7} {\TFarrow}}  #8}

\newcommand {\TFaseq} [7]
    {#2 \TFiosep  #3 {\TFarrow{#4 \quad #5}}_{#6}\:  #7}

\newcommand {\TFaGseq} [6]
    {#1 \TFiosep  #2 {\TFarrow{#3 \quad #4}}_{#5}\:  #6}

\newcommand{\maptag}[3]{\mathrel{\merge(#1,#2,#3)}}


\newcommand {\TFFpsystem} {\mathcal{TF}\lightning}


% --------------------------------------
% --- CLF
% --------------------------------------

% a sequent calculus presentation of CLF
% Note: proof-terms dropped

\newcommand {\CLFsystem} {CLF}

%--- expression checking (EC)

\newcommand{\CLFECarrow}{\xleftarrow}

% #1= intuitionistic context  #2= linear context  #3= sync type
\newcommand{\CLFECseq}[3]
      {#1 ; #2 \CLFECarrow{} #3}

%--- normal object cheking (NC)

\newcommand{\CLFNCarrow}{\xLeftarrow}

% #1= intuitionistic context  #2= linear context  #3= type
\newcommand{\CLFNCseq}[3]
      {#1 ; #2 \CLFNCarrow{} #3}

%--- monadic object cheking (MC)

\newcommand{\CLFMCarrow}{\xLeftarrow}

% #1= intuitionistic context  #2= linear context  #3= sync type
\newcommand{\CLFMCseq}[3]
      {#1 ; #2 \CLFMCarrow{} #3}

%--- pattern expansion (PE)

\newcommand{\CLFPEarrow}{\xleftarrow}

% #1= intuitionistic context  #2= linear context  #3= patterns  #4= async type
\newcommand{\CLFPEseq}[4]
      {#1 ; #2 \, \vert \, #3 \CLFPEarrow{} #4}

%--- atomic object inference (AI)

\newcommand{\CLFAIarrow}{\xRightarrow}

% #1= intuitionistic context  #2= linear context  #3= async type
\newcommand{\CLFAIseq}[3]
      {#1 ; #2 \CLFAIarrow{} #3}

%--- forward chaining

\newcommand{\CLFFCarrow}{\Downarrow}

\newcommand{\commit}{\mathbf{commit}}

% #1= intuitionistic context  #2= linear context  #3= async type #4= monadic type
\newcommand{\CLFFCseq}[4]
      {#1 ; #2 \, \vert \, #3 \CLFFCarrow  #4}


% --------------------------------------
% --- CLF IO_\top
% --------------------------------------

% an IO_\top presentation of CLF
% Note: proof-terms dropped

\newcommand {\CLFIOsystem} {CLF_{IO_{\top}}}

%--- expression checking (EC)

% #1= intuitionistic context #2= input #3= output #4= top flag  #5= sync type
\newcommand{\CLFIOECseq}[5]
      {#1 ; #2 \IOsep #3 \CLFECarrow[#4]{} #5}

%--- normal object cheking (NC)

% #1= intuitionistic context #2= input #3= output #4= top flag  #5= type
\newcommand{\CLFIONCseq}[5]
      {#1 ; #2 \IOsep #3 \CLFNCarrow[#4]{} #5}

%--- monadic object cheking (MC)

% #1= intuitionistic context #2= input #3= output #4= top flag  #5= sync type
\newcommand{\CLFIOMCseq}[5]
      {#1 ; #2 \IOsep #3 \CLFMCarrow[#4]{} #5}

%--- pattern expansion (PE)

% #1= intuitionistic context #2= input #3= output  #4= patterns #5= top flag #6= async type
\newcommand{\CLFIOPEseq}[6]
      {#1 ; #2 \IOsep #3 \, \vert \, #4 \CLFPEarrow[#5]{} #6}

%--- atomic object inference (AI)

% #1= intuitionistic context #2= input #3= output #4= top flag  #5= async type
\newcommand{\CLFIOAIseq}[5]
      {#1 ; #2 \IOsep #3 \CLFAIarrow[#4]{} #5}

%--- forward chaining

% #1= intuitionistic context  #2= input #3= output  #4= async type #5= top flag   #6= monadic type
\newcommand{\CLFIOFCseq}[6]
      {#1 ; #2 \IOsep #3 \, \vert \, #4 \CLFFCarrow_{#5}  #6}


% --------------------------------------
% --- CLF Frames
% --------------------------------------

% a Frames presentation of CLF
% Note: proof-terms dropped

\newcommand {\CLFFRsystem} {CLF_{\mathcal{F}}}

%--- expression checking (EC)

% #1= intuitionistic context #2= strict #3= input #4= output #5= top flag  #6= sync type
\newcommand{\CLFFRECseq}[6]
      {#1 ; #2 ; #3 \IOsep #4 \CLFECarrow[#5]{} #6}

%--- normal object cheking (NC)

% #1= intuitionistic context #2= strict #3= input #4= output #5= top flag  #6= type
\newcommand{\CLFFRNCseq}[6]
      {#1 ; #2 ; #3 \IOsep #4 \CLFNCarrow[#5]{} #6}

%--- monadic object cheking (MC)

% #1= intuitionistic context #2= stritc #3= input #4= output #5= top flag  #6= sync type
\newcommand{\CLFFRMCseq}[6]
      {#1 ; #2 ; #3 \IOsep #4 \CLFMCarrow[#5]{} #6}

%--- atomic object inference (AI)

% #1= intuitionistic context #2= strict #3= input #4= output #5= top flag  #6= async type
\newcommand{\CLFFRAIseq}[6]
      {#1 ; #2 ; #3 \IOsep #4 \CLFAIarrow[#5]{} #6}

%--- pattern expansion (PE)

% #1= intuitionistic context #2= strict #3= input #4= output  #5= patterns #6= top flag #7= async type
\newcommand{\CLFFRPEseq}[7]
      {#1 ; #2 ; #3 \IOsep #4 \, \vert \, #5 \CLFPEarrow[#6]{} #7}


%--- forward chaining

% #1= intuitionistic context  #2= strict #3= input #4= output  #5= async type #6= top flag   #7= monadic type
\newcommand{\CLFFRFCseq}[7]
      {#1 ; #2 ; #3 \IOsep #4 \, \vert \, #5 \CLFFCarrow_{#6}  #7}


% --------------------------------------
% ---  CLF Tag-Frame-Fast
% --------------------------------------

% a Tag-Frame-Fast presentation of CLF
% Note: proof-terms and intuitionistic context dropped

\newcommand {\CLFTFsystem} {CLF_{TF}}

%--- expression checking (EC)

%  #1=Stack id  #2=Alias in   #3=Alias out   #4=Input  #5=Output
%  #6= strict tag   #7=delete tag  #8=top flag #9= goal
\newcommand {\CLFTFECseq}[9]
    {#1; \; #2 \TFiosep #3; \; #4 \TFiosep #5  {\CLFECarrow[#8]{#6 \quad #7}}  #9}

%--- normal object cheking (NC)

%  #1=Stack id  #2=Alias in   #3=Alias out   #4=Input  #5=Output
%  #6= strict tag   #7=delete tag  #8=top flag #9= goal
\newcommand {\CLFTFNCseq}[9]
    {#1; \; #2 \TFiosep #3; \; #4 \TFiosep #5  {\CLFNCarrow[#8]{#6 \quad #7}} #9}

%--- monadic object cheking (MC)

%  #1=Stack id  #2=Alias in   #3=Alias out   #4=Input  #5=Output
%  #6= strict tag   #7=delete tag  #8=top flag #9= goal
\newcommand {\CLFTFMCseq}[9]
    {#1; \; #2 \TFiosep #3; \; #4 \TFiosep #5  {\CLFMCarrow[#8]{#6 \quad #7}} #9}

%--- pattern expansion (PE)

% 10 arguments, must split

% this is to write the rest of parameters
\newcommand {\CLFTFPErest} [1] {#1}

%  #1=Stack id  #2=Alias in   #3=Alias out   #4=Input  #5=Output  #6= Patterns
%  #7= strict tag   #8=delete tag  #9=top flag #10= goal
\newcommand {\CLFTFPEseq}[9]
    {#1; \; #2 \TFiosep #3; \; #4 \TFiosep #5 \; \vert #6  {\CLFPEarrow[#9]{#7 \quad #8}} \CLFTFPErest}


% --------------------------------------
% ---  Extra stuff (mostly Jeff's)
% --------------------------------------


\def\emptypf{\emptyset}
\def\bnfas{\mathrel{::=}}
\def\bnfalt{\mid}
\def\S{\sigma}
\def\St{\Sigma}
\def\D{\Delta}
\def\L{\alpha}
\def\tt{\mathrm{t}\!\mathrm{t}}
\def\ff{\mathrm{f}\!\mathrm{f}}

\def\none{\blacklozenge}

\newcommand{\dom}[1]{\mathrm{dom}(#1)}
\newcommand{\cod}[1]{\mathrm{cod}(#1)}

\newcommand{\hs}[1]{\hspace*{#1em}}

\newcommand {\rsq}[4]{#2 \gg #3 \TFiosep #4}


\def\above#1#2{\begin{array}[b]{c}\relax #1\\ \relax #2\end{array}}
\def\derI{\mathcal{D}_1}
\def\derII{\mathcal{D}_2}

\newcommand{\IOTseq}[4]{#1\TFiosep #2\:\Longrightarrow_{#3}\: #4}
\newcommand{\Fseq}[5]{#1;#2\TFiosep #3 \longrightarrow_{#4} \: #5 }


%%% Local Variables: 
%%% mode: plain-tex
%%% TeX-master: "E:\\lolli_clf\\opsem"
%%% End: 


\newcommand{\piIn}[2]{\overline{#1}\langle #2 \rangle} 
\def\ppar{\mathbin{|}}
\def\lra{\longrightarrow}

% Fix dependence on mathtools --kw
%\usepackage{mathtools}
\let\xleftarrow\leftarrow
\let\xrightarrow\rightarrow
\let\xLeftarrow\Leftarrow
\let\xRightarrow\Rightarrow

% Fix hyphenation problems --kw
\emergencystretch1.5em

% No page numbers in camera ready version

% Show page numbers
%\def\thepage{\arabic{page}}
%\advance\voffset by -0.5in

% Comment out long spans of text
\long\def\junkstartshere#1\junkendshere{}

% British spelling gotchas:
%   judgement -> judgment
%   arguement -> argument
%   focussing -> focusing
%   focussed -> focused
%   modelling -> modeling
%   modelled -> modeled
%     (but note
%        occurred, occurring,
%        discussed, discussing,
%        compelled, compelling
%      in both British and American
%      -- stress on 2nd syllable doubles consonant in American)

\title{Monadic Concurrent Linear Logic Programming}

\numberofauthors{4}

\author{ 
\alignauthor Pablo L\'{o}pez \\
\affaddr{Universidad de M\'{a}laga} % \\
\email{\affaddr lopez@lcc.uma.es}
\alignauthor Frank Pfenning \\
% \affaddr \hbox{Computer Science Department} %\\
\affaddr Carnegie Mellon University %\\
\email{\affaddr fp@cs.cmu.edu}
\alignauthor Jeff Polakow \\
\affaddr{AIST, CVS, JST}
\titlenote{
National Institute of Advanced Industrial Science and Technology (AIST),
Research Center for Verification and Semantics (CVS),
CREST, Japan Science and Technology Agency (JST)
\newline
\newline
This research has been supported by the Office of Naval Research (ONR)
under grant MURI N00014-04-1-0724: Distributed System Security via
Logical Frameworks.
}\\[1.25ex]
\email{\affaddr j-polakow@aist.go.jp}
\alignauthor Kevin Watkins \\
% \affaddr \hbox{Computer Science Department} %\\
\affaddr Carnegie Mellon University %\\
\email{\affaddr kw@cs.cmu.edu}
}

\begin{document}

\conferenceinfo{PPDP'05,} {July 11--13, 2005, Lisbon, Portugal.}
\CopyrightYear{2005}
\crdata{1-59593-090-6/05/0007}

\maketitle

\begin{abstract}
Lolli is a logic programming language based on the asynchronous
propositions of intuitionistic linear logic.  It uses a backward
chaining, backtracking operational semantics.  In this paper we extend
Lolli with the remaining connectives of intuitionistic linear logic
restric\-ted to occur inside a monad, an idea taken from the concurrent
logical framework (CLF).  The resulting language, called LolliMon, has a
natural forward chaining, committed choice operational semantics inside
the monad, while retaining Lolli's semantics outside the monad.
LolliMon thereby cleanly integrates both concurrency and saturation with
logic programming search.  We illustrate its expressive power through
several examples including an implementation of the pi-calculus, a
call-by-need lambda-calculus, and several saturating algorithms presented
in logical form.
\end{abstract}

\category{D.1.6}{Logic Programming}{}
\category{D.1.3}{Concurrent Programming}{}
\terms Languages, Theory
\keywords linear logic, committed choice concurrent logic programming, operational semantics

%\vspace{1mm}
%\noindent
%{\bf Categories and Subject Descriptors:} 
%D.1.6 {[{\bf Logic Programming}]}; D.1.3 {[{\bf Concurrent Programming}]}
%
%\vspace{1mm}
%\noindent
%{\bf General Terms:} Languages
%
%\vspace{1mm}
%\noindent
%{\bf Keywords:} linear logic, committed choice concurrency, operational semantics

\section{Introduction}
\label{sec:intro}

Combining computational paradigms, such as functional, logic,
imperative, concurrent, constraint, or ob\-ject-oriented programming, in
a clean, uniform, and effective way is generally quite difficult
because of the deep philosophical, theoretical, and pragmatic
differences that divide them.  Nevertheless, there have been many
attempts to do so.  One reason is that many algorithms and
computational patterns that can be expressed naturally in one paradigm
may be quite cumbersome in others.  By combining paradigms we can hope
to get the best of both worlds.  An additional reason is that
applications demand increasingly richer programming concepts.  In
particular, concurrent and distributed programming are becoming more
prevalent, and therefore corresponding facilities are available in
modern languages, many of which were not originally conceived with
concurrency in mind.  This is reflected in the difficulties we have in
reasoning about programs written in these languages, especially when
they employ concurrency using low-level primitives.

In this paper we advance the thesis that \emph{logic} can be a
powerful unifying force in the design of multi-paradigm languages.
More concretely, we extend back\-ward-chaining logic programming with
\emph{concurrency} and \emph{saturation}, employing logical concepts
such as lax truth and linearity in novel ways to obtain an elegant,
minimalistic, yet general account of these computational phenomena.

One starting point for our design is the linear logic programming
language Lolli~\cite{Hodas94ic}.  It is based on a fragment of
intuitionistic linear logic~\cite{Girard87tcs} endowed with an
operational semantics in the form of backward-chaining search.  It forms
an \emph{abstract logic programming language}~\cite{Miller91apal} and
therefore permits a natural interpretation of the logical connectives as
goal-directed search instructions.  In Andreoli's
terminology~\cite{Andreoli92} we say that Lolli is based on the
\emph{asynchronous connectives} of linear logic.  Lolli programs use
linear assumptions to encode state and thereby capture some
aspects of imperative computation in a logical manner.

The second starting point is the concurrent logical framework
CLF~\cite{Cervesato02tr,Watkins02tr,Watkins04types}.  The aspect of
CLF most relevant to this paper is that it extends the asynchronous
fragment of intuitionistic linear logic with syn\-chro\-nous
connectives, but restricts them to occur inside a
\emph{monad}~\cite{Moggi91}.  From a logical perspective, this monad
is a modal operator satisfying the laws of lax
logic~\cite{Fairtlough97ic} in its judgmental
formulation~\cite{Pfenning01mscs}.
The present paper introduces a computational perspective:
the monad represents a dividing line between
the backward-chaining, backtracking semantics with asynchronous
connectives outside the monad, and forward-chaining, committed choice
semantics with synchronous connectives inside the monad.

The monadic encapsulation prevents undesirable interference between
the two forms of computation and permits their harmonious
co-existence.  As we will see, the two forms of computation are
closely coupled and mutually dependent, yet predictable and clear\-ly
identifiable without destroying the useful properties of their
components.  It seems difficult if not impossible to achieve this
without the use of a monad.  Monads have been employed for similar
reasons in the context of functional programming~\cite{Wadler92popl}.
In logic programming, their only use we are aware of is for
higher-order programming with data structures~\cite{Bekkers95ilps,McGrail97},
which is quite different from our application.

There have been many other studies of concurrent logic programming (see,
for example, an early survey \cite{Shapiro89}), and concurrent logic
programming in fragments of linear logic (for example, LO~\cite{Andreoli90oopsla} and
ACL~\cite{Kobayashi94tr}), but we are not aware of
any that combine several forms of search in the same orthogonal way.
Perhaps most closely related is Forum~\cite{Miller96tcs}, which is based on
classical linear logic, but to the authors' knowledge its operational
semantics and the interaction between concurrent and sequential
computation in Forum has never been fully clarified.  Another notable
attempt combining backward and forward chaining by Harland et
al.\ \cite{Harland00pstt} is an interesting foundational study of proof
search, but it is not clear whether it can be realized in a practical
logic programming language.
Another item of related work is by Bozzano et al.~\cite{Bozzano04tplp},
who study a bottom-up logic programming semantics for a fragment of
classical linear logic in which weakening is admissible.  The proof
construction procedure is oriented toward complete forward reasoning
rather than a committed choice non-deterministic operational semantics
and is therefore more suited for model-checking rather than concurrent
programming.  Futhermore, it does not include any backward search.  On
the other hand it incorporates an aspect of saturation which is
significantly more general than ours in that it allows a whole set of
affine states to act as a fixpoint, terminating if any transition
remains in the same set.  It would be interesting to consider if their
ideas could be applied in our setting to reason about the behavior of
LolliMon programs in a similar manner.

Our proposal has been implemented as a concrete language we call
\emph{LolliMon}.  We have programmed numerous running examples,
several of which are given in this paper.  From these examples one can
extract certain basic programming techniques, although we do not yet
claim to have full understanding of the programming methodology or
implementation challenges posed by our language.

One aspect that seems to be emerging is a close relationship between
concurrent and saturating computation.  The latter has been popular
for some time in the treatment of constraints (see, for example,
CHR~\cite{Fruehwirth98jlp}) and recently in the logical specification
and complexity analysis of several
algorithms~\cite{Ganzinger01ijcar,Mcallester02jacm}.  Most recently,
the deletion of facts has been introduced into the logical algorithm
framework. This appears to be modeled well by the consumption of
linear assumptions in LolliMon.

The remainder of the paper is structured as follows.
Section~\ref{sec:logical-rules} introduces the syntax, judgments, and
  rules of the logic on which LolliMon is based.
Section~\ref{sec:opsem} describes the operational semantics of LolliMon.
Section~\ref{sec:examples} presents several examples of LolliMon logic
  programs.
Section~\ref{sec:conclusion} summarizes and sketches future work.

\section{Logical Rules}
\label{sec:logical-rules}

In this section
 we describe the basic syntactic structure of LolliMon,
the logical rules on which its proof search strategy is based, and the
basic aspects of the operational semantics that controls the application of
the rules during proof search. Section~\ref{sec:opsem} treats the design
of the operational semantics and the interesting phenomena that arise
in modeling concurrency and saturation more completely.

\subsection{Syntax}
The formula language of LolliMon consists of \emph{asynchronous}
connectives~$A$, which it shares with the earlier logic programming
language Lolli, and \emph{synchronous} connectives~$S$, which were not
a part of Lolli's formula language.  The distinction between
synchronous and asynchronous connectives is fundamental; as we will
see, the proof rules for asynchronous connectives have a natural logic
programming interpretation in terms of backwards, goal-directed search
(in the style pioneered by the original Prolog and formalized by the
notion of \emph{uniform proofs}~\cite{Miller91apal}), while the
synchronous connectives have a logic programming interpretation based
on undirected, forward-chaining, concurrent execution.

The connection between these two rather different modes of execution is
mediated by the monad constructor~$\{\cdot\}$. Syntactically, the monad
constructor allows synchronous connectives to be embedded within
asynchronous formulas. Semantically, the point at which the monad
constructor is introduced in a proof demarcates a transition from
backward to forward reasoning (thinking, as usual, of the construction
of the proof from the outermost goal upward).

The formula language of LolliMon is as follows.
$$
\begin{array}{lll}
A & ::= &  P 
      \mid \top 
      \mid A_1 \with A_2 
      \mid \\&& A_1 \limp A_2 
      \mid A_1 \iimp A_2 
      \mid \forall x{:}\tau.A  
      \mid \{ S \} 
\\[1ex]
S & ::= &  A
      \mid \bang A
      \mid \one 
      \mid S_1 \tensor S_2 
      \mid \exists x{:}\tau.S 
\end{array}
$$

As in other logic programming languages such as
$\lambda$Prolog~\cite{Miller86iclp}, object types~$\tau$ are used to
enforce the well-formedness of the objects of the system being
modeled.
The syntax of objects and object types is not discussed in the present
paper; it is an entirely standard (prenex) polymorphically typed
$\lambda$-calculus.  The nonterminal~$P$ stands for an atomic proposition,
which must be well-formed according to the discipline of the object type system.

In the presentation of our logic, we will make use of three kinds of
formula context, $\Gamma$, $\Delta$, and $\Psi$, which respectively contain unrestricted
(intuitionistic) asynchronous hypotheses, linear asynchronous
hypotheses, and linear synchronous hypotheses.
$$
\begin{array}{lll@{\qquad}l}
  \Gamma & ::= & \cdot \mid \Gamma, A & \mbox{Unrestricted context}\\
  \Delta & ::= & \cdot \mid \Delta, A & \mbox{Linear context}\\
  \Psi   & ::= & \cdot \mid S, \Psi & \mbox{Synchronous context}\\
\end{array}
$$
For the purposes of this description we may take $\Gamma$ and
$\Delta$ to be multisets. We write $\Delta_1,\Delta_2$ to
denote the multiset union of $\Delta_1$ and $\Delta_2$.
There will be no explicit contraction or weakening rules for~$\Gamma$,
but contraction and weakening for~$\Gamma$ are metatheorems of LolliMon's
sequent calculus.

In the
LolliMon interpreter, the order of hypotheses is significant only in
terms of which hypothesis is non-deterministically selected for
focusing.  However the third sort of context, $\Psi$, is not only linear but
ordered; the ordering of $\Psi$ helps to control the sequence of
invertible left rules applied to its elements, as we will see below.

% Right inversion
\def\CLFRIseq#1#2#3{#1;#2\Rightarrow#3}
% Left focusing
\def\CLFLFseq#1#2#3#4{#1;#2;#3\mathrel{>\!\! >}#4}
% (L)ax forward chaining
\def\CLFLseq#1#2#3{#1;#2\rightarrow#3}
% (L)ax left focusing
\def\CLFLLFseq#1#2#3#4{#1;#2;#3\mathrel>#4}
% Left inversion
\def\CLFLIseq#1#2#3#4{#1;#2;#3\rightarrow#4}
% Right focusing
\def\CLFRFseq#1#2#3{#1;#2\mathrel{>\!\! >}#3}

We present the logical rules of LolliMon in a sequent calculus
formulation intended to bring out most clearly the relationship of the
rules to LolliMon's operational semantics.  For a
natural-deduction-style presentation, see the technical report on
CLF (the dependent type theory of which LolliMon's
logic is a fragment)~\cite{Watkins02tr}.  

We classify the logical rules of LolliMon into three basic kinds: {\it
inversion}, {\it focusing}, and {\it transition}.  Inversion rules are
always applied eagerly; their use can never cause proof search to fail.
Focusing rules are those which entail a significant choice; their use
can cause a particular branch in proof search to fail.  All inversion
and focusing rules act either on a specific hypothesis (left) or on the
conclusion (right) of the sequent, and we identify them accordingly.
Lastly, we have transition rules that initiate or terminate a sequence
of inversion or focusing rules.

Very roughly, the operational semantics of LolliMon can be described
in terms of five modes of execution.
Four capture inversion and focusing, each on the left and the right.
An additional mode is needed because of differences in the way
left focusing happens depending on whether the formula on the right
is in the monad (a synchronous formula $S$) or outside it (an atomic
formula $P$).

Our first two computation modes, \emph{right inversion} and
\emph{left focusing}, are associated with asynchronous
formulas $A$, and are inherited from the operational semantics of
Lolli~\cite{Hodas94ic}.
$$
\begin{array}{ll}
  \CLFRIseq{\Gamma}{\Delta}{A} & \hbox{Right inversion} \\
  \CLFLFseq{\Gamma}{\Delta}{A}{P} & \hbox{Left focusing} \\
\end{array}
$$
%
The other three computation modes, \emph{monadic left focusing},
\emph{left inversion}, and \emph{right focusing}, are
associated with synchronous formulas $S$.  In addition, we
have a transition sequent, which, when read bottom-up,
marks the end of left inversion and precedes
monadic left or right focusing.  We call this a
\emph{forward chaining} sequent in view of its eventual
operational interpretation.  The transition on the other side, from right
inversion to left focusing, takes place at $\CLFRIseq{\Gamma}{\Delta}{P}$.
There is no separate sequent form for the latter transition, because there
only one kind of focusing is possible.
$$
\begin{array}{ll}
  \CLFLseq{\Gamma}{\Delta}{S} & \hbox{Forward chaining} \\
  \CLFLLFseq{\Gamma}{\Delta}{A}{S} & \hbox{Monadic left focusing} \\
  \CLFLIseq{\Gamma}{\Delta}{\Psi}{S} & \hbox{Left inversion} \\
  \CLFRFseq{\Gamma}{\Delta}{S} & \hbox{Right focusing} \\
\end{array}
$$
The operational meaning of these modes is discussed below, along
with the logical rules for each of them.

\subsection{Asynchronous Formulas}
\label{ssec:async-form}
As LolliMon is a conservative extension of Lolli, which is based on
the asynchronous connectives $\top$, $\with$, $\iimp$, $\limp$, and $\forall$,
the logical rules for these
connectives are inherited essentially unchanged from Lolli.  For each
connective, there is a right inversion rule and a left focusing rule.  The
basic search strategy decomposes the right formula using inversion rules
until it is atomic, then non-determinis\-tically selects a
hypothesis, which is focused on until the same atomic
formula is reached.  The following rules are used to select an
unrestricted or a linear hypothesis for focusing.
$$
\nsrule{\CLFLFseq{\Gamma,A}{\Delta}{A}{P}}
       {\CLFRIseq{\Gamma,A}{\Delta}{P}}
       {\GRuleName{\CLFsystem}{\hbox{uhyp}}}
\qquad
\nsrule{\CLFLFseq{\Gamma}{\Delta}{A}{P}}
       {\CLFRIseq{\Gamma}{\Delta,A}{P}}
       {\GRuleName{\CLFsystem}{\hbox{lhyp}}}
$$

The following axiom rule says we are finished if the atomic formula from
the goal matches the atomic formula obtained by focusing on a
hypothesis.
$$
\nsrule{}
       {\CLFLFseq{\Gamma}{\cdot}{P}{P}}
       {\GRuleName{\CLFsystem}{\hbox{atm}}}
$$

The right and left rules for unrestricted implication are standard.
$$
\nsrule{\CLFRIseq{\Gamma,A}{\Delta}{B}}
       {\CLFRIseq{\Gamma}{\Delta}{A \iimp B}}
       {\GRuleName{\CLFsystem}{\iimp_R}}
\quad
\nsrule{\CLFLFseq{\Gamma}{\Delta}{B}{P}
        \andalso
        \CLFRIseq{\Gamma}{\cdot}{A}}
       {\CLFLFseq{\Gamma}{\Delta}{A\iimp B}{P}}
       {\GRuleName{\CLFsystem}{\iimp_L}}
$$
%
Note the restriction in $\iimp_L$ that the derivation of $A$ not
require linear hypotheses.

The right and left rules for linear implication are also standard:
$$
\nsrule{\CLFRIseq{\Gamma}{\Delta,A}{B}}
       {\CLFRIseq{\Gamma}{\Delta}{A\limp B}}
       {\GRuleName{\CLFsystem}{\limp_R}}
\quad
\nsrule{\CLFLFseq{\Gamma}{\Delta_1}{B}{P}
        \quad
        \CLFRIseq{\Gamma}{\Delta_2}{A}}
       {\CLFLFseq{\Gamma}{\Delta_1,\Delta_2}{A\limp B}{P}}
       {\GRuleName{\CLFsystem}{\limp_L}}
$$

The right and left rules for additive conjunction and unit are
as follows:
$$
\nsrule{}
       {\CLFRIseq{\Gamma}{\Delta}{\top}}
       {\GRuleName{\CLFsystem}{\top_R}}
\qquad
\hbox{(no $\top_L$ rule)}
$$
$$
\nsrule{\CLFRIseq{\Gamma}{\Delta}{A}
        \andalso
        \CLFRIseq{\Gamma}{\Delta}{B}}
       {\CLFRIseq{\Gamma}{\Delta}{A\with B}}
       {\GRuleName{\CLFsystem}{\with_R}}
$$
$$
\nsrule{\CLFLFseq{\Gamma}{\Delta}{A}{P}}
       {\CLFLFseq{\Gamma}{\Delta}{A\with B}{P}}
       {\GRuleName{\CLFsystem}{\with_{L1}}}
\qquad
\nsrule{\CLFLFseq{\Gamma}{\Delta}{B}{P}}
       {\CLFLFseq{\Gamma}{\Delta}{A\with B}{P}}
       {\GRuleName{\CLFsystem}{\with_{L2}}}
$$

Finally, we have right and left rules for universal quantification.
$$
\nsrule{\CLFRIseq{\Gamma}{\Delta}{[a/x]A}}
       {\CLFRIseq{\Gamma}{\Delta}{\forall x{:}\tau.A}}
       {\GRuleName{\CLFsystem}{\forall_R}}
\qquad
\nsrule{\CLFLFseq{\Gamma}{\Delta}{[t/x]A}{P}}
       {\CLFLFseq{\Gamma}{\Delta}{\forall x{:}\tau.A}{P}}
       {\GRuleName{\CLFsystem}{\forall_L}}
$$
%
Here, $a$~stands for a fresh parameter of type~$\tau$, the scope of
which is restricted to the subderivation above the~$\forall_R$ rule. The term~$t$
in the corresponding left rule must be well-formed of type~$\tau$
with respect to all parameters that are in scope at the point the
left rule is applied.  At the level of description considered here,
the term~$t$ is picked non-deterministically.  In the LolliMon
implementation, it is determined by unification.

This
concludes the description of the fragment of LolliMon inherited from
Lolli.

There is one additional rule associated with right inversion:
the right rule for the monad constructor $\{\cdot\}$.
$$
\nsrule{\CLFLseq{\Gamma}{\Delta}{S}}
       {\CLFRIseq{\Gamma}{\Delta}{\{S\}}}
       {\GRuleName{\CLFsystem}{\{\}_R}}
$$
%
This rule moves the proof search procedure from the goal-directed
inversion mode of Lolli to the forward-chaining
mode of LolliMon.  The logical interpretation of the forward
chaining judgment appearing in the premise is the lax judgment of lax
logic~\cite{Fairtlough97ic,Pfenning01mscs}.  Because the monad
constructor connects the truth judgment and the lax judgment of the
logic, it is subject to somewhat different symmetries than the rest of
the asynchronous connectives.  In particular, its left rule does not
appear among the left focusing rules treated in this section;
instead, it appears in the monadic left focusing rules
discussed in Section~\ref{ssec:sync-form}.

\subsection{Synchronous Formulas}
\label{ssec:sync-form}

We now turn to the proof rules concerning the synchronous connectives
$\tensor$, $\one$, $\exists$, and $\bang$.

Here, the right rules are not invertible, so the strategy of first
decomposing the formula on the right by inversion is inadequate.
In general, the derivation of a sequent with a
synchronous goal formula~$S$ must first allow for some forward
reasoning from hypotheses before applying non-invertible right rules
to break down the goal $S$; see Andreoli's analysis of linear logic
proof search~\cite{Andreoli92} for more details.  Exactly which
hypotheses are focused on for forward reasoning is a
non-deterministic choice.

We have the following two rules, which allow for
focusing on an unrestricted or linear hypothesis.
$$
\nsrule{\CLFLLFseq{\Gamma,A}{\Delta}{A}{S}}
       {\CLFLseq{\Gamma,A}{\Delta}{S}}
       {\GRuleName{\CLFsystem}{\hbox{uhyp}'}}
\qquad
\nsrule{\CLFLLFseq{\Gamma}{\Delta}{A}{S}}
       {\CLFLseq{\Gamma}{\Delta,A}{S}}
       {\GRuleName{\CLFsystem}{\hbox{lhyp}'}}
$$
%
Here the special nature of the lax judgment underlying the forward
chaining mode enters: the hypothesis selected must have a head of the
form $\{S'\}$, instead of a head of the form $P$.  This requirement is
made explicit in the $\{\}_L$ rule below.  There is additional
non-determinism here, compared with the case of heads of the
form~$P$, because the formula~$S'$ need not match the ultimate goal
formula~$S$ on the right.

The monad constructor $\{S'\}$ is decomposed by the following left
rule.
$$
\nsrule{\CLFLIseq{\Gamma}{\Delta}{S'}{S}}
       {\CLFLLFseq{\Gamma}{\Delta}{\{S'\}}{S}}
       {\GRuleName{\CLFsystem}{\{\}_L}}
$$
After removing the monad constructor, the formula $S'$ is decomposed
by left rules, which \emph{are} invertible for synchronous formulas.
These left rules can add further parameters and hypotheses to the
contexts.  Finally, once $S'$ is completely decomposed, we return
to the forward chaining judgment:
$$
\nsrule{\CLFLseq{\Gamma}{\Delta}{S}}
       {\CLFLIseq{\Gamma}{\Delta}{\cdot}{S}}
       {\GRuleName{\CLFsystem}{\rightarrow\rightarrow}}
$$

Thus, the overall effect of the sequence of rules beginning with
$\hbox{uhyp}'$ or $\hbox{lhyp}'$ and ending with
$\rightarrow\rightarrow$ is to consume some hypotheses during left focusing
by propagating them to subgoals,
and introduce others by left inversion. The goal
formula~$S$ on the right is left alone.  
We think of this sequence as an atomic step in the execution of a concurrent
system. The application of the $\rightarrow\rightarrow$ rule
demarcates the atomic step. As described in Section~\ref{sec:opsem},
the use of the $\rightarrow\rightarrow$ rule forces a commitment to
all of the suspended non-deterministic choices encountered during the
atomic step.

We have seen how the forward chaining judgment allows working on
the left by reasoning forward from hypotheses.
The other way to proceed, when forward chaining, is to choose to
work on the goal~$S$ on the right, by applying non-invertible right
rules.  Operationally, the decision to focus
on the goal on the right implicitly terminates forward chaining mode.
We capture this behavior with the following rule.
$$
\nsrule{\CLFRFseq{\Gamma}{\Delta}{S}}
       {\CLFLseq{\Gamma}{\Delta}{S}}
       {\GRuleName{\CLFsystem}{>\!\! >\rightarrow}}
$$
This rule corresponds to the coercion from the truth judgment to the
lax judgment in lax logic.

Note that all three forward chaining rules, $\hbox{uhyp}'$,
$\hbox{lhyp}'$, and $>\!\! >\rightarrow$, are applicable whenever
the search procedure is in forward chaining mode.  For the purposes of
the logical description of LolliMon, we are content to leave as a
non-deterministic choice which forward chaining rule should be
applied.  The operational semantics described in
Section~\ref{sec:opsem} imposes further structure on these choices.

\subsubsection{Monadic Left Focusing}
The rules associated with the monadic left focusing judgment
$\CLFLLFseq{\Gamma}{\Delta}{A}{S}$ are very similar to those described
previously for $\CLFLFseq{\Gamma}{\Delta}{A}{P}$.  In both cases,
non-invertible left rules are applied to the formula~$A$ until its
head ($\{S'\}$ or $P'$, respectively) is reached.  However, the two
sequent forms differ, in that the head~$P'$ of the latter is forced to
be equal to the atomic goal~$P$ produced by right inversion,
while the head~$\{S'\}$ of the
former does not have to be equal to the formula~$S$ on the
right.

Here we show the modified left rule for linear implication; the other
left rules for asynchronous connectives are modified along the same
lines.
$$
\nsrule{\CLFLLFseq{\Gamma}{\Delta_2}{B}{S}
        \andalso
        \CLFRIseq{\Gamma}{\Delta_1}{A}}
       {\CLFLLFseq{\Gamma}{\Delta_1,\Delta_2}{A\limp B}{S}}
       {\GRuleName{\CLFsystem}{\limp'_L}}
$$

\subsubsection{Left Inversion}
The rules for invertibly decomposing a synchronous formula on the left
are simple adaptations of the usual sequent calculus left rules for
each connective to our setting.  When inversion begins, the context~$\Psi$
contains a single formula~$S'$.  The context~$\Psi$ can contain more than
one formula after some left rules have been applied.  Invertibility allows
us to force the decomposition to occur in a left-to-right order.  This
avoids permutations of the left rules that would otherwise lead to an
unwanted explosion in the number of proofs.  Thus, each rule decomposes
the leftmost synchronous formula in the context~$\Psi$.  When an asynchronous
formula becomes leftmost, it is moved to the linear context, because there are
no more invertible left rules to apply to it.
Finally, at some point the context~$\Psi$ becomes empty, and forward chaining
resumes.

The left rules for multiplicative conjunction and unit are as follows:
$$
\nsrule{\CLFLIseq{\Gamma}{\Delta}{\Psi}{S}}
       {\CLFLIseq{\Gamma}{\Delta}{\one,\Psi}{S}}
       {\GRuleName{\CLFsystem}{\one_L}}
\qquad
\nsrule{\CLFLIseq{\Gamma}{\Delta}{S_1,S_2,\Psi}{S}}
       {\CLFLIseq{\Gamma}{\Delta}{S_1\tensor S_2,\Psi}{S}}
       {\GRuleName{\CLFsystem}{\tensor_L}}
$$

The left rule for the existential quantifier is the following:
$$
\nsrule{\CLFLIseq{\Gamma}{\Delta}{[a/x]S',\Psi}{S}}
       {\CLFLIseq{\Gamma}{\Delta}{\exists x{:}\tau.S',\Psi}{S}}
       {\GRuleName{\CLFsystem}{\exists_L}}
$$
%
As in the $\forall_R$ rule, the parameter $a$ has type $\tau$ and its
scope is the subderivation above the $\exists_L$ rule.

The left rule for the unrestricted modality adds a hypothesis to the
unrestricted context.  The syntax of LolliMon requires the formula underneath
the modality to be asynchronous.
$$
\nsrule{\CLFLIseq{\Gamma,A}{\Delta}{\Psi}{S}}
       {\CLFLIseq{\Gamma}{\Delta}{\bang A,\Psi}{S}}
       {\GRuleName{\CLFsystem}{\bang_L}}
$$

Finally, the rule for moving an asynchronous formula to the linear context
is as follows:
$$
\nsrule{\CLFLIseq{\Gamma}{\Delta,A}{\Psi}{S}}
       {\CLFLIseq{\Gamma}{\Delta}{A,\Psi}{S}}
       {\GRuleName{\CLFsystem}{\hbox{async}}}
$$


\subsubsection{Right Focusing}
At some point the derivation transitions from forward chaining to
right focusing mode, which applies non-invertible right rules to search for a proof of the
synchronous goal formula.  Each of these rules is a simple adaptation
of the corresponding right rule from linear logic.

The right rules for multiplicative conjunction and unit are as follows:
$$
\nsrule{}
       {\CLFRFseq{\Gamma}{\cdot}{\one}}
       {\GRuleName{\CLFsystem}{\one_R}}
\qquad
\nsrule{\CLFRFseq{\Gamma}{\Delta_1}{S_1}
        \andalso
        \CLFRFseq{\Gamma}{\Delta_2}{S_2}}
       {\CLFRFseq{\Gamma}{\Delta_1,\Delta_2}{S_1\tensor S_2}}
       {\GRuleName{\CLFsystem}{\tensor_R}}
$$

The right rule for the existential quantifier is the following:
$$
\nsrule{\CLFRFseq{\Gamma}{\Delta}{[t/x]S}}
       {\CLFRFseq{\Gamma}{\Delta}{\exists x{:}\tau.S}}
       {\GRuleName{\CLFsystem}{\exists_R}}
$$
As in the~$\forall_L$ rule, we require $t$ to have type $\tau$ and determine
it by unification.

The right rule for the unrestricted modality is the following:
$$
\nsrule{\CLFRIseq{\Gamma}{\cdot}{A}}
       {\CLFRFseq{\Gamma}{\cdot}{\bang A}}
       {\GRuleName{\CLFsystem}{\bang_R}}
$$
%
The syntax of LolliMon requires an asynchronous formula under the
unrestricted modality, so at the point the right rule for the modality
is applied, the right focusing stage ends.

Finally, when an asynchronous formula appears as a subformula of a
synchronous formula, the right focusing stage ends, and a new
right inversion stage begins.
$$
\nsrule{\CLFRIseq{\Gamma}{\Delta}{A}}
       {\CLFRFseq{\Gamma}{\Delta}{A}}
       {\GRuleName{\CLFsystem}{\Rightarrow>\!\!>}}
$$

The logical rules of LolliMon enjoy the admissibility of
cut.  This is a consequence of a corresponding theorem
for the natural deduction formulation of CLF~\cite{Watkins02tr}.


\section{Operational Semantics}
\label{sec:opsem}
Designing an operational semantics for the proof rules given in
Section~\ref{sec:logical-rules} is non-trivial.  There is a great deal of
non-determinism to be removed, and there are many choices to be taken,
potentially leading to different operational behavior.  In this section, we informally
describe an operational semantics for LolliMon which results in a
language integrating both committed choice concurrency and
backtracking goal-directed search.

A number of design decisions guided our specification.  Since LolliMon
is conceived as a concurrent logic programming language, completeness
of the proof search procedure is not a concern.  More concretely,
committed choice and quiescence (to be described below) lead to a
strategy that will not explore all of the search space.  In addition,
the operational semantics must be compatible with Lolli's over the fragment
that Lolli and LolliMon share.

The operational semantics of LolliMon is a bidirectional proof search
procedure.  When LolliMon proofs contain no monadic goals, the proof
search procedure is similar to that of other logic languages such as
$\lambda$Prolog or Lolli: a backward-chaining, backtracking
search for a uniform proof.  In contrast, when a monadic
goal is found, a forward-chaining, committed choice strategy is
applied.  Thus the monad affects the basic mode of proof search, and
can be interpreted as staging the LolliMon computation.  In the
remainder of this section we detail the different stages of this
computation: goal-directed search, backward-chaining and
forward-chaining.

\subsection{Example: Asynchronous $\pi$-Calculus}

In order to make the discussion of LolliMon's operational semantics
more concrete, we will be using a representation of an
asynchronous\footnote{The use of ``asynchronous'' here is not related
to our earlier uses of the term.}  variant of Milner's
$\pi$-calculus~\cite{Milner99} as a running example. The $\pi$-calculus is
based on a syntax for concurrently executing \emph{processes}~$P$. The
processes interact by sending and receiving messages along named
\emph{channels}~$c$.

The basic process constructors are parallel composition $P\mid Q$ and
its unit~$0$ (representing a process that
is finished).  There is also a construct $\nu c. P(c)$ for binding a
new channel name within a process.  In traditional presentations
of the $\pi$-calculus, scope extrusion is used to propagate the bound
channel name from its initial scope to more global scopes.  In LolliMon,
we will see that fresh parameters introduced by the~$\exists_L$ rule
fulfill the same function.

\def\pin#1#2{#1(#2)}
\def\pout#1#2{\bar{#1}\langle#2\rangle}
\def\prin#1#2{!#1(#2)}

The process constructors concerned with communication are $\pin{c}{x}.P(x)$ for
inputting a value along the channel~$c$ (with the process~$P(x)$
continuing to execute using the communicated value in place of
the variable~$x$), and $\pout{c}{v}$
for outputting the value~$v$ along the channel~$c$. In this simple
variant of the $\pi$-calculus, the only values are the channel names
themselves.

In addition, there is a replicated variant $\prin{c}{x}.P(x)$ of the
input construction, which allows arbitrarily many values to be
received along channel~$c$, each spawning a new copy of~$P(x)$ with
the received value subsituted for~$x$. The replicated input process
itself persists forever, once it has been introduced.

\def\rep#1{\ulcorner#1\urcorner}
\def\kwd#1{\expandafter\def\csname k#1\endcsname{\textbf{#1}}}
\kwd{type}\kwd{expr}\kwd{chan}\kwd{par}\kwd{zero}\kwd{new}
\kwd{in}\kwd{rin}\kwd{out}
\def\to{\rightarrow}

This concludes the syntax of the $\pi$-calculus. In the LolliMon
implementation, each of these process constructors is represented by a
term constructor in LolliMon's term language.
$$
\begin{array}{l}
  \kexpr : \ktype. \qquad  \kchan : \ktype. \\[1ex]

  \kpar : \kexpr \to \kexpr \to \kexpr. \\
  \kzero : \kexpr. \\
  \knew : (\kchan \to \kexpr) \to \kexpr. \\
\end{array}$$ $$\begin{array}{l}
  \kin : \kchan \to (\kchan \to \kexpr) \to \kexpr. \\
  \krin : \kchan \to (\kchan \to \kexpr) \to \kexpr. \\
  \kout : \kchan \to \kchan \to \kexpr.
\end{array}
$$
%
Note the use of higher-order abstract syntax to represent the
$\pi$-calculus binding constructs.  There is a bijective
\emph{representation function}~$\rep{\cdot}$ mapping $\pi$-calculus
process expressions to well-typed, canonical LolliMon terms having
type~$\kexpr$.
$$
\begin{array}{l}
  \rep{P\mid Q} = \kpar\,\rep{P}\,\rep{Q} \\
  \rep{0} = \kzero \\
  \rep{\nu c.P(c)} = \knew\,(\lambda c{:}\kchan.\rep{P(c)}) \\
  \rep{\pin{c}{x}.P(x)} = \kin\,c\,(\lambda x{:}\kchan.\rep{P(x)}) \\
  \rep{\prin{c}{x}.P(x)} = \krin\,c\,(\lambda x{:}\kchan.\rep{P(x)}) \\
  \rep{\pout{c}{v}} = \kout\,c\,v
\end{array}
$$

\subsection{Goal-Directed Proof Search}
\label{ssec:goal-directed}
The right inversion rules of LolliMon are exactly those of Lolli, except for
$\GRuleName{\CLFsystem} {\{\}_R}$.  These rules are applied from the
bottom up to decompose a goal into atomic and/or monadic subgoals. It
should be noted that rules other than the right rules are only
considered when the goal is either atomic or monadic.  When there are
no monadic goals involved, LolliMon proofs are
uniform~\cite{Miller91apal}.  When there are monadic goals, this
strategy can be seen as an extension of uniform proofs, where the
monad introduces an intermediate stage, forward chaining, before
the goal is further decomposed. Forward chaining, which corresponds
to the lax judgment of lax logic, is particular to LolliMon.

The non-determinism implicit in the choice of a term $t$ in the
$\exists_R$ rule (as well as the $\forall_L$ rule) is handled as usual
by generating a logic variable and relying upon unification to
appropriately instantiate this variable.  Since our term calculus is a
$\lambda$-calculus, we use a deterministic version of higher-order
unification that suspends any non-pattern~\cite{Miller91jlc} unification
problems until they are further instantiated.  Any suspended
unification problems remaining at the end of a query are reported as
leftover constraints.

The non-deterministic splitting of the linear context implicit in the
$\tensor_R$, $\limp_L$, and $\limp'_L$ rules
is common to traditional linear logic programming
languages~\cite{Hodas94ic,PymHarland94}.  The low-level management of
linear hypotheses turns out to be unaffected by the addition of a
forward chaining phase to proof search.  Our strategy for managing
linear hypotheses is a new variation of the I/O system for
Lolli~\cite{Hodas94ic}.  The new strategy is described in detail
in~\cite{LopPol05lpar}.

\subsection{Backward Chaining}
\label{ssec:bc}
When an atomic goal is encountered in right inversion mode,
a hypothesis with an atomic head must be chosen to focus on.
Following the behavior of Lolli, our implementation keeps track of the
order in which hypotheses\footnote{We think of program clauses as
  hypotheses assumed at the beginning of each query.}, both linear and
unrestricted, were assumed and tries to focus first on the most recent
ones whose heads unify with the atomic goal; see
Section~\ref{ssec:ordering} for more discussion on clause ordering in
LolliMon.  The usual Prolog backtracking semantics is used when a
particular choice fails, or to search for alternative solutions.

Upon choosing a hypothesis, the system switches to left
focusing mode and applies left rules to the selected
hypothesis, keeping track of any new subgoals generated (by the
$\iimp_L$ and $\limp_L$ rules) along the way.  When the head of the
selected hypothesis is finally exposed, it is unified with the
original atomic goal, and unless unification fails, proof search will
continue by solving any pending subgoals.  This strategy is the usual
backchaining mechanism of Prolog.

If left focusing encounters the $\with$ connective, 
two rules, $\with_{L1}$ and $\with_{L2}$, might be applied.
The resolution of this choice is made
by the same mechanism that chooses a hypothesis to focus on in the
first place, as though there were two clauses, one for each branch
of the $\with$.

\subsection{Forward Chaining}
\label{ssec:forward}
As mentioned in Section~\ref{ssec:async-form}, in order to construct
a proof of $\{S\}$, the LolliMon system enters a forward chaining
mode and tries to derive $\CLFLseq{\Gamma}{\Delta}{S}$.
The most important difference
between goal-directed search (as in Lolli) and forward chaining is
that for forward chaining, the formula~$S$ on the right does not have
to match the head of a clause selected for focusing on the left.  In
fact, it is perfectly acceptable for a program clause to be selected
that doesn't mention~$S$ at all.  However, only program clauses having
\emph{some} monadic formula $\{S'\}$ at their heads can be selected.

As Section~\ref{ssec:sync-form} hints, we think of the forward
chaining phase of proof search as a sequence of atomic steps.  Each
atomic step begins with an application of the $\hbox{uhyp}'$ or
$\hbox{lhyp}'$ rule that selects a hypothesis~$A$ to focus on, leading to
a sequent of the form
$$
  \CLFLLFseq{\Gamma}{\Delta}{A}{S}.
$$
%
Next, the system breaks down $A$ by applying a series of non-invertible
left rules, postponing any new subgoals in the manner described in
Section~\ref{ssec:bc}.  Eventually, the head of $A$, which
must be a formula of the form~$\{S'\}$, is reached:
$$
\CLFLLFseq{\Gamma}{\Delta'}{\{S'\}}{S}.
$$
%
Here $\Delta'$ is the subset of $\Delta$ remaining after the
context is split between this branch of the proof and any subgoals.
The LolliMon implementation determines $\Delta'$ lazily.

At this point, after establishing that the chosen clause is indeed
monadic, any pending subgoals are solved.  If a
subgoal cannot be solved, then the system backtracks to the beginning
of the forward chaining cycle and looks for a new clause to focus on.
The procedure up to this point is the same as that
employed for backward chaining, except that no unification is involved.

Next, supposing any pending subgoals have been solved, the $\{\}_L$ rule is applied, leading to
$$
  \CLFLIseq{\Gamma}{\Delta'}{S'}{S},
$$
and after applying a series of invertible left rules to~$S'$, we
have a sequent of the form
$$
  \CLFLIseq{\Gamma,\Gamma_0}{\Delta',\Delta_0}{\cdot}{S}
$$
%
where $\Gamma_0$ and $\Delta_0$ are any new assumptions added during the
decomposition of $S'$.  Finally, the rule $\rightarrow\rightarrow$ is
applied to finish the atomic step, and we are left with
$$
\CLFLseq{\Gamma,\Gamma_0}{\Delta',\Delta_0}{S},
$$
%
which is again in the forward chaining mode.  The application of
$\rightarrow\rightarrow$ additionally causes the system to commit to
this atomic step by removing all the choice points generated while
focusing on the original $A$ or solving subgoals generated during focusing.

Thus, the overall effect of the atomic step is to possibly consume some
linear hypotheses and possibly introduce new unrestricted or linear
hypotheses. In addition, the process of proving the atomic step may
have had a side effect on the state of the unification store.

The other way to proceed when in forward chaining mode
$$
  \CLFLseq{\Gamma}{\Delta}{S}
$$
is to focus on the right-hand formula~$S$ using the $>\!\!>\rightarrow$ rule,
leading to
$$
  \CLFRFseq{\Gamma}{\Delta}{S}.
$$ 
%
This terminates the forward chaining phase, and the LolliMon
interpreter proceeds by applying non-invertible right rules to~$S$.
Sections \ref{ssec:quiescence}~and~\ref{ssec:saturation} describe
strategies for
deciding when to end forward chaining in this way.

\kwd{o}\kwd{proc}\kwd{msg}

\textbf{EXAMPLE.}\enspace 
Using the syntax introduced earlier for the $\pi$-calculus, we can set
up a forward-chaining interpreter for $\pi$-calculus processes.
First, we introduce the following predicates:
$$
\begin{array}{l}
\kproc : \kexpr \to \ko. \\
\kmsg : \kchan \to \kchan \to \ko. \\
\end{array}
$$
These two predicates correspond to the message and process types of
Kobayashi and Yonezawa's ACL~\cite{Kobayashi94tr}.
The idea of the interpreter is that the state of the $\pi$-calculus
execution will be represented by the multiset of linear hypotheses
available in forward chaining mode.
The goal on the right-hand side while the interpreter is running will
always be~$\one$.
Suppose we want to initialize
the interpreter with a process
$$
\begin{array}{rll}
  P_0 &=& \rep{\pout{c}{x} \mid \pin{c}{y}.0} \\
      &=& \kpar\,(\kout\,c\,x)\,(\kin\,c\,(\lambda y.\kzero)). \\
\end{array}
$$
Initially, we have only the linear hypothesis $\kproc\,P_0$ in
the context.  By adding the rule
$$
  A_0 = \forall P.\forall Q.\kproc\,(\kpar\,P\,Q) \limp \{\kproc\,P \tensor \kproc\,Q\}
$$
to our logic program, we enable a forward chaining step that consumes
the hypothesis $\kproc\,P_0$ and introduces two new hypotheses, namely
$\kproc\,(\kout\,c\,x)$ and $\kproc\,(\kin\,c\,(\lambda y.\kzero))$.

Suppose, then, that we start with just~$P_0$ running, which according
to the invariant of the interpreter, means that we are searching for
a LolliMon proof of the sequent
$$
  \CLFLseq{\Gamma}{\kproc\,P_0}{\one}.
$$
Here $\Gamma$~includes all the rules of our logic program for the
interpreter.
The first step is to focus on the program clause $A_0$ above,
using the $\hbox{uhyp}'$ rule, leaving
$$
  \CLFLLFseq{\Gamma}{\kproc\,P_0}{A_0}{\one}.
$$
At this point we need to apply the rule $\forall'_L$ twice to get
rid of the universal quantifiers at the outside of~$A_0$.  The LolliMon
system will instantiate the bound variables $P$~and~$Q$ of the universal
quantifiers by unification, but we know what the
instantiation will ultimately be, namely $P=\kout\,c\,x$ and
$Q=\kin\,c\,(\lambda y.\kzero)$, so for this discussion we use it from the start.

We are then left with the sequent
$$
  \CLFLLFseq{\Gamma}{\kproc\,P_0}{\kproc\,(\kpar\,P\,Q)
    \limp \{\kproc\,P \tensor \kproc\,Q\}}{\one}
$$
with $P$~and~$Q$ as above. The next step is to use $\limp'_L$ to
eliminate the linear implication.  This leaves us with
$$
  \CLFLLFseq{\Gamma}{\cdot}{\{\kproc\,P \tensor \kproc\,Q\}}{\one}\\
$$ 
%
while the second premise
$$
  \CLFRIseq{\Gamma}{\kproc\,P_0}{\kproc\,(\kpar\,P\,Q)}
$$
%
is stored on the subgoal stack.  Here we've also anticipated the
partitioning the linear hypotheses in the only way that
can succeed: namely, the single linear hypothesis $\kproc\,P_0$ is
allocated to the second subgoal.

Since we now have formula of the form $\{S'\}$ on the left, we suspend
left focusing and apply inversion
to the stored subgoal:
$$
  \CLFRIseq{\Gamma}{\kproc\,P_0}{\kproc\,(\kpar\,P\,Q)},
$$
%
which is quickly dealt with, since $P_0=\kpar\,P\,Q$.  

We then return to the main focusing proof.  The only rule that can be
used is $\{\}_L$, which leads to
$$
  \CLFLIseq{\Gamma}{\cdot}{\kproc\,P\tensor\kproc\,Q}{\one}.
$$
At this point, several left inversion rules
are applied, and we end up with
$$
  \CLFLIseq{\Gamma}{\kproc\,P, \kproc\,Q}{\cdot}{\one}.
$$
%
Now the only applicable rule is $\rightarrow\rightarrow$, which leaves us with
$$
  \CLFLseq{\Gamma}{\kproc\,P, \kproc\,Q}{\one}
$$
%
to prove.  There were no non-deterministic choices involved in the
execution of this atomic step, because our program had only one
clause, but in general, there might be choice points created between
the use of $\hbox{uhyp}'$ or $\hbox{lhyp}'$ and the use of
$\rightarrow\rightarrow$.

It is perhaps worth noting at this point that we are not at all
interested in the question of \emph{completeness}; that is, the
possibility of finding any proof admitted by the logical rules.  Such
a notion of completeness might be useful in model checking, or some
other analysis intended to explore the entire state space of the
concurrent system.  But LolliMon is not a model checker; the aim here
is to \emph{execute} a concurrent system, not to map its entire state
space.  Accordingly, just after each forward step is taken, any choice
points introduced during the the proof of the forward step are
dropped; LolliMon \emph{commits} to the forward step. Also, once the
decision to stop forward chaining has been taken, LolliMon commits to
that decision, never going back to consider executions that might have
lasted longer.

\subsection{Quiescence}
\label{ssec:quiescence}
We have seen how LolliMon takes each atomic step in forward chaining
mode.  There remain the issues of \emph{which} forward chaining steps
it should take, and \emph{when} to stop taking forward chaining steps
and return to a goal-directed search strategy.
The underlying logic on which LolliMon is based does not constrain
these choices at all.
In this section, and the following, we discuss the termination criterion,
the decision about when to finish forward chaining and return to goal-directed
proof search.  Section~\ref{ssec:ordering} treats the question of which
forward steps should be taken.

Since the intended semantics of forward chaining mode is the simulation
of various concurrent object systems, the decision about when to finish
forward chaining is equivalent to deciding how to terminate a concurrent
system.  The strategies which turn out to be most useful in practice are
based purely on the behavior of the concurrent system; that is, they are
based only on what forward steps are available to be taken.
One might also
imagine a criterion based on the goal formula~$S$ that will be proved on
the right once forward chaining ends, but this turns out to be fraught
with difficulties.  The current LolliMon implementation never
looks at the formula on the right when in forward chaining mode, either
to decide which forward steps should be taken, or to decide when to
stop taking forward steps.  This reduces further the space of possible
proofs found by the LolliMon interpreter.

The simplest strategy 
depending only on what
forward steps are available to be taken
is \emph{quiescence}: the concurrent execution (that is, the forward
chaining phase) ends when \emph{no} forward step is available to be
taken.

{\bf EXAMPLE.}\enspace
We can illustrate the notion of quiescence by completing the simple
$\pi$-calculus execution begun above. First, we reveal the remaining
clauses in the $\pi$-calculus interpreter's logic program:
$$
\begin{array}{l}
  \kproc\,\kzero \limp \{\one\}. \\
  \forall P.\kproc\,(\knew\,(\lambda c.P\,c)) \limp \{\exists c. \kproc\,(P\,c)\}. \\[1ex]
  \forall C.\forall V.\kproc\,(\kout\,C\,V) \limp \{\kmsg\,C\,V\}. \\
  \forall C.\forall P.\kproc\,(\kin\,C\,(\lambda x.P\,x)) \\
    \qquad{}\limp \{\forall V.\kmsg\,C\,V \limp \{\kproc\,(P\,V)\}\,\}. \\
  \forall C.\forall P.\kproc\,(\krin\,C\,(\lambda x.P\,x)) \\
    \qquad{}\limp \{\bang\forall V.\kmsg\,C\,V \limp \{\kproc\,(P\,V)\}\,\}. \\
\end{array}
$$
The first two clauses are similar to the clause we have already seen
for processes of the form $\kpar\,P\,Q$.  In each case a process is
interpreted by LolliMon's own logical connectives---concurrent composition
becomes $\tensor$, the unit $\kzero$ of composition becomes $\one$, and
the channel-binding construct becomes $\exists$.  In the last case,
the $\exists_L$ rule will introduce a fresh parameter to stand
for the name of the new channel.

The final three clauses are concerned with asynchronous communication.
Each message containing the value $V$ and flowing along a channel $C$
is represented by a linear hypothesis of the form $\kmsg\,C\,V$. The
clause for $\kout$ creates such messages.  The clauses for $\kin$ and
$\krin$ use the powerful technique of \emph{forward-generating new
clauses}. In this case, the clause
$$
  \forall V.\kmsg\,C\,V \limp \{\kproc\,(P\,V)\}
$$
is introduced as a new linear (for $\kin$) or unrestricted (for $\krin$)
hypothesis.

Continuing the example execution of Section~\ref{ssec:forward}, the
sequent
$$
  \CLFLseq{\Gamma}{\kproc\,(\kout\,c\,x), \kproc\,(\kin\,c\,(\lambda y.\kzero))}{\one},
$$
after several atomic steps, will eventually reach
$$
  \CLFLseq{\Gamma}{\kmsg\,c\,x, (\forall V.\kmsg\,c\,V \limp \{\kproc\,\kzero\})}{\one}.
$$
%
At this point, none of the clauses in $\Gamma$ can be successfully
focused on, so the only choice is to use the second linear hypothesis.
This leads to
$$
  \CLFLLFseq{\Gamma}{\kmsg\,c\,x}{\forall V.\kmsg\,c\,V \limp \{\kproc\,\kzero\}}{\one},
$$
%
and the atomic step is finished by applying left rules and solving
the subgoal $\kmsg\,c\,x$ using the other linear hypothesis.
At this point we have
$$
  \CLFLseq{\Gamma}{\kproc\,\kzero}{\one},
$$
and finally, after one more atomic step, we have
$$
  \CLFLseq{\Gamma}{\cdot}{\one},
$$
at which point no atomic forward chaining step is possible, so quiescence
has been reached. Accordingly, the LolliMon system ends the forward chaining
phase by applying $>\!\! >\rightarrow$, and the proof is completed with $1_R$.

\subsection{Saturation}
\label{ssec:saturation}
The last section introduced quiescence, defined as a state in which
no further forward steps are possible.  But sometimes it is more useful
to consider \emph{saturation}: a state in which forward steps may be
possible, but they do not lead to any new information.

\kwd{r}\kwd{rr} 
%
For example, we may want to execute \emph{bottom-up logic programs}
that repeatedly apply forward reasoning steps to a set of facts,
each step extending the set of known facts, until no further facts
can be deduced from those that are already known.

{\bf EXAMPLE.}\enspace
A simple example is a program to compute the
transitive closure of a finite relation. Suppose we have a finite
number of unrestricted hypotheses of the form $\kr\,A\,B$ for various
pairs $A$~and~$B$.  We can then generate the transitive closure
$\krr\,A\,B$ by the following program.
$$
\begin{array}{l}
  \forall A.\forall B.\kr\,A\,B \iimp \{!\krr\,A\,B\}. \\
  \forall A.\forall B.\forall C.
    \krr\,A\,B \iimp \krr\,B\,C \iimp \{!\krr\,A\,C\}. \\
\end{array}
$$

Each forward step taken by this program uses some facts concerning
$\kr$ and $\krr$ to derive a new fact about $\krr$ and add it to the
unrestricted context.  If we ran the program under the quiescence
criterion for termination, it would execute forever (assuming we
start with at least one fact concerning $\kr$), because more forward
steps can always be taken, even if they only regenerate facts that
were already known.

The saturation criterion, on the other hand, disallows forward
reasoning steps that either have no effect on the set of available
linear and unrestricted hypotheses, or simply reintroduce unrestricted
hypotheses that were already known.  We define a \emph{non-trivial
  step} to be an atomic forward chaining step which causes a change to
either of the logical contexts, or to the state of the unification
store.  We then define saturation to mean that no non-trivial step is
possible.

Assuming there is at least one axiom concerning $\kr$, the program above
saturates in a finite number of steps, because all the hypotheses
introduced are unrestricted and $\krr$ is finite.  Once every
deducible consequence $\krr\,A\,B$ ends up in the unrestricted context,
no non-trivial steps are possible, and saturation is reached.  We exhibit
realistic examples of this bottom-up technique in
Section~\ref{sec:examples}.

The LolliMon implementation always uses saturation to decide when to
stop forward chaining.  This is not observably different from quiescence,
except that a concurrent execution might run forever under quiescence
(after some point always returning to the same state), while terminating
under saturation.

\subsection{Clause Ordering}
\label{ssec:ordering}
Saturation checking is implemented by
term indexing.  Specifically, LolliMon unifies unrestricted and linear
hypotheses, distinguished by
appropriate tags, in one context, which is implemented as a discrimination
tree~\cite{RSV01}.  The discrimination tree allows the
system to efficiently (i.e.\ without scanning the entire context) check
whether a given formula is already in the context, and thus whether a
step is non-trivial. However, it also complicates keeping track of clause
order.  

As specified in Section~\ref{ssec:bc}, during backchaining the
system will try predicate clauses, starting from
the most recently assumed, and working back to the least recent.
Such behavior, though standard in traditional logic
programming languages, is too limiting for some natural LolliMon
programs, such as our $\pi$-calculus encoding or the meta-circular
interpreter.\footnote{Available with the LolliMon distribution.}  To this
end, LolliMon has another directive, \texttt{\#{}fair}, which
declares the programmer's intent to have the clauses of a particular
predicate chosen fairly, rather than in a fixed order. The
implementation approximates fairness by trying clauses in a random
order.

Including the idea of fair choice is important when we want to be
able to simulate a concurrent system in such a way that any given
run of the LolliMon program will give us a ``typical'' execution,
rather than some ``special'' execution.  For example, in the $\pi$-calculus
interpreter described above, the fair-choice model allows both
possible outcomes, $P$~and~$Q$, for the process
$$
  \nu c.\pout{c}{c} \mid (\pin{c}{x}.P) \mid (\pin{c}{x}.Q),
$$
whereas the fixed-order mode
would
rule out one or the other of $P$~and~$Q$.

When in the monadic left focusing judgment, any clauses having heads
of the form $\{S\}$, rather than an atomic predicate, are eligible
to be selected.  The implementation
always chooses from among these clauses in a fair way. Only for atomic
predicates $P$ is the fixed-order mode available.

\section{Examples}
\label{sec:examples}
In this section we present three concrete examples of LolliMon
specifications.  We use \texttt{typewriter} font for LolliMon code.
All of these examples, and many more, are bundled with the prototype
distribution of LolliMon.\footnote{\texttt{http://www.cs.cmu.edu/\~{}fp/lollimon/}}

We summarize LolliMon concrete syntax as follows:
$$
\begin{array}{r@{{}={}}l@{\qquad}r@{{}={}}l@{\qquad}r@{{}={}}l@{\qquad}r@{{}={}}l}
\limp & \verb@-o@
&
\iimp & \verb@=>@
&
\tensor & \verb@,@
&
\with & \verb@&@
\\
\{\, \} & \verb@{ }@
&
\bang & \verb@!@
&
\top & \verb@top@
&
\one & \verb@one@
\end{array}
$$ 
Additionally we have
$$
\begin{array}{r@{\;\;=\;\;}l}
\lambda x{:}\tau.M
&
\verb@x \ M@
\\
\forall x{:}\tau.A
&
\verb@pi x \ A@
\\
\exists x{:}\tau.S
&
\verb@sigma x \ S@
\end{array}
$$ 
where \verb@M@, \verb@A@ and \verb@S@ are the concrete representations
of $M$, $A$ and $S$. The type of bound variables \verb@x@ is
automatically inferred by the system. This is always possible, since
all constants are declared and types are restricted to prenex
polymorphism.  

LolliMon also provides several built-in predicates and terms including
lists (\verb@nil@  for empty list and \verb@::@ for cons),
basic integer arithmetic (\verb@is@, \verb@+@, etc.),
and output (\verb@write@ and \verb@nl@).

We allow clauses of the form \verb"(A, B) -o {C}", which stands
for \verb"B -o (A -o {C})", and, in a similar manner,
\verb"(A, B) => {C}", which stands for \verb"B => (A => {C})".
Because of the subgoal selection strategy,
both of these will first match \verb"A", then \verb"B", and
then commit the forward chaining step.

We also use the standard convention that (unbound) uppercase letters
represent logic variables that are implicitly universally quantified
at the outermost level.

\subsection{CKY Parsing}
\label{ssec:cky}
We present a LolliMon specification of a CKY parser based on
McAllester's logical algorithm~\cite{Mcallester02jacm}.
This example only uses the
unrestricted context and relies upon forward chaining and saturation.

The main predicate is \texttt{parse X I J} which asserts
that the input substring from position $i$ to position $j$, inclusive, is of
syntactic category $X$.

Starting from an input list,
we assert \texttt{s}$\,i\,c$ for each character of input $c$ at
position $i$.  When the list becomes empty, we initiate a forward
chaining computation, as indicated by the monad brackets in the first
clause below.  When the forward chaining computation saturates, we
test if the whole string (which goes from 1 to $n$) is of category $S$
and succeed if that is so.

\begin{small}\begin{verbatim}
  load N nil S <= {parse S 1 N}.
  load I (C::Cs) S <= 
    J is I+1, (s J C => load J Cs S).

  start Cs S <= load 0 Cs S.
\end{verbatim}\end{small}

Grammar productions are in normal form of either
$X \rightarrow c$ for a character $c$ or
$X \rightarrow Y\; Z$ for non-terminals $Y$ and $Z$.
These are represented in the form \texttt{rule X (char C)}
and \texttt{rule X (jux Y Z)}, respectively.  We
just run these grammar rules from right to left,
for example asserting we have an $X$ from $i$ to $k$ if
we already know that the substring from $i$ to $j$
is a $Y$ and the substring from $j+1$ to $k$ is a $Z$.

\begin{small}\begin{verbatim}
  rule X (char C), s I C => {!parse X I I}.

  rule X (jux Y Z), parse Y I J, J' is J+1, 
    parse Z J' K => {!parse X I K}.
\end{verbatim}\end{small}

This concludes the parser.  Note that it does
not use the linear context and relies entirely on
saturation for termination.

\subsection{Call-by-Name and Futures}
\label{ssec:mml}

We present a LolliMon implementation of a $\lambda$-calculus with a
call-by-name semantics, then add futures, a simple construct for
parallel evaluation~\cite{Halstead85}.  This core can be extended to
functional languages with features such as recursion, polymorphism,
mutable reference, continuations, exceptions, concurrency in the style
of Concurrent ML, and distributed computation.  Some of these can be
found with the prototype distribution of LolliMon or as sample specifications in
an earlier report on CLF~\cite{Cervesato02tr}.

Our presentation uses \emph{linear des\-ti\-na\-tion-pass\-ing
style} \cite{Pfenning04aplas}, which is based on three main syntactic
categories: \emph{expressions} to be evaluated, \emph{frames} for
suspended computations, and \emph{destinations} for values.

Destinations are special in that they are just
abstract names and have no further structure.  They will
be represented as parameters in our implementation.

Corresponding to the three syntactic categories we have three
predicates: \texttt{eval} evaluates an expression, given a destination,
\texttt{comp} captures a suspended computation together with the
eventual destination of its result, and \texttt{return} records the
value returned to a destination.

Unlike the usual techniques from (higher-order) logic programming
languages, these predicates will take on their appropriate operational
meaning in the \emph{linear context}, and we execute using
forward chaining with only trivial backward chaining to match atomic
formulas.  We start with a linear context containing only
\texttt{eval $e$ $d_0$}
for an expression $e$ to be evaluated and an initial destination $d_0$
and reach quiescence in a state with
\texttt{return $v$ $d_0$}
for a value $v$.  In this example, values are just expressions, although
in some examples it is clearer to separate them out in their own
syntactic class.

As a first simple example, consider call-by-name evaluation.
We represent expressions using higher-order abstract syntax.

\begin{small}\begin{verbatim}
  lam : (exp -> exp) -> exp.
  app : exp -> exp -> exp.
\end{verbatim}\end{small}

A $\lambda$-abstraction is a value, so we return it directly
to the given destination $D$.

\begin{small}\begin{verbatim}
  eval (lam x \ E x) D
  -o {return (lam x \ E x) D}. 
\end{verbatim}\end{small}
\noindent The right-hand side of the linear implication is a monadic formula,
which enforces that this rule is used only during forward chaining.
This will be true for all other rules in this implementation,
except for one at the top level.

An application is evaluated by first evaluating the function,
while a frame waits for the evaluation's result.
The result must be passed in a fresh destination $d_1$, which is
a parameter created by the $\exists_L$ rule.

\begin{small}\begin{verbatim}
  eval (app E1 E2) D
  -o {sigma d1 \ eval E1 d1,
         comp (app1 d1 E2) D}.
\end{verbatim}\end{small}
\noindent
The destination \verb"D" of the new frame is the same as the original
destination for the application.

Finally, when evaluation of the function is complete, we can
substitute the argument into the function body and evaluate
the result.  This substitution is modeled by meta-level application,
a standard technique with higher-order abstract syntax.
Here we need to coordinate \emph{two} linear
assumptions: one returns a value, while the other is the waiting
frame.

\begin{small}\begin{verbatim}
  return (lam x \ E1' x) D1,
  comp (app1 D1 E2) D
  -o {eval (E1' E2) D}.
\end{verbatim}\end{small}

This completes the main program.  In order to invoke it
at the top level, we have a predicate \verb"evaluate"
which evaluates an expression $e$ and prints its value.
To print the result we use the built-in predicates \verb"write"
and \verb"nl", the latter starting a new line.

\begin{small}\begin{verbatim}
  evaluate E
    o- (pi d0 \ eval E d0
           -o {sigma V \ return V d0,
                  write V, nl}).
\end{verbatim}\end{small}

A query \verb"evaluate" $e$ now starts with backward chaining (note the
direction \verb"o-" of the outermost implication), creating the initial
destination $d_0$ and linear assumption \verb"eval" $e$ $d_0$ before
entering the monad and initiating forward chaining.  We continue to
forward chain until we reach quiescence, which is the case precisely
when we are returning a value to the initial destination, or the Mini-ML
computation gets stuck.  This cannot happen for a well-typed Mini-ML
term, but we have elided the necessary
type-checking predicate here, so it is possible for an \verb"evaluate"
query to fail.

As a generalization of the above we consider \emph{futures} derived from
Multilisp~\cite{Halstead85}: the construct \verb"future" $e$ immediately
returns with a \emph{promise} and starts a new thread evaluating $e$ in
parallel.  If the promise is ever evaluated (touched) we block until the
evaluation of $e$ completes, which will give us the value of the
promise.

The most straightforward implementation in linear destination-passing
style uses new expression forms \verb"future" $e$ and \verb"promise" $p$ where
$p$ is a destination.  We also use a new frame \verb"future1" $d_1$ to
await the completion of the future. 
Evaluating \verb"future" $e$ then creates two
new destinations: one, called $d_1$, to stand for the value of 
$e$ directly, and one, called $p$, to represent the promise.

\begin{small}\begin{verbatim}
  eval (future E) D
  -o {sigma d1 \ sigma p \ 
        eval E d1, comp (future1 d1) p,
        return (promise p) D}.
\end{verbatim}\end{small}
\noindent
If evaluation of $E$ completes with value $V$, we install $V$
as the value of the future $p$.  This must occur in
the unrestricted context: the promise $p$ may actually be
replicated many times via substitution, or it may never be needed at
all, violating linearity in both cases.

\begin{small}\begin{verbatim}
  return V D1,
  comp (future1 D1) P
  -o {!return V P}.
\end{verbatim}\end{small}
\noindent
If we ever evaluate (touch) a promise, we create a frame waiting
for its value.

\begin{small}\begin{verbatim}
  eval (promise P) D
  -o {comp (promise1 P) D}.
\end{verbatim}\end{small}
\noindent
Finally, if the value of the promise is both available and needed,
we pass it to the proper destination.

\begin{small}\begin{verbatim}
  !return V1 P,
  comp (promise1 P) D
  -o {return V1 D}.
\end{verbatim}\end{small}

Thus, futures can be explained in four rules, without affecting the rules
for functions and applications previously introduced.  Such semantic
modularity is a an elegant property of linear destination-passing style,
and it is immediately reflected in the LolliMon encoding.

\subsection{Checking Graph Bipartiteness}
\label{ssec:gbc}
We now present a LolliMon specification of graph bipartiteness
checking based on the logical algorithms in the formulation
of Ganzinger and McAllester~\cite{Ganzinger02iclp}.  

A graph is represented by a set of unrestricted assumptions
\texttt{edge} $u$ $v$ for nodes $u$ and $v$, indicating an edge
from $u$ to $v$.  We also start with a \emph{linear} assumption
\texttt{unlabeled} $u$ for every node $u$.  We further have constant
labels \texttt{a} and \texttt{b} to represent two partitions.

The outer loop of the algorithm picks an arbitrary unlabeled node,
labels it (arbitrarily) with \texttt{a}, and then propagates
information by labeling all neighbors of \texttt{a} nodes with
\texttt{b} and vice versa.  When this propagation saturates, we check if
there are any nodes labeled both \texttt{a} and \texttt{b}.  If so, the
graph cannot be bipartite.  If not, we pick another unlabeled node, label
it with \texttt{a} and repeat the propagation step.  If there are no
unlabeled nodes left we know the graph is bipartite.

First, the top-level iteration.  In order to cut off unnecessary
iterations, we check for a node labeled both \texttt{a} and \texttt{b}
first (note use of the \verb@#ordered@ directive).  So
\texttt{iterate} succeeds if the graph is \emph{not} bipartite.

\begin{small}\begin{verbatim}
  #ordered iterate.
  iterate o- sigma U \
    !labeled U a, !labeled U b, top.
  iterate o- sigma U \
    unlabeled U, (labeled U a => {iterate}).
\end{verbatim}\end{small}

\noindent By the structure of the second clause,
the overall computation consists of a variable number of
stages, each of which is a computation run to saturation.
Assumptions of the form \texttt{labeled} $u$ are always
unrestricted,
indicated by the form \texttt{A => \{B\}} rather than linear
implication \texttt{A -o \{B\}}.  \texttt{unlabeled} assumptions are
always linear, so the second line consumes one.  We use
\texttt{top} in the first clause so we can stop iterating
even if there are unlabeled nodes left.

The next three rules are for label propagation, including
one to remove linear assumptions \texttt{unlabeled U} in case
the node \texttt{U} was labeled during the saturation process.

\begin{small}\begin{verbatim}
  !labeled U a, !edge U V -o {!labeled V b}.
  !labeled U b, !edge U V -o {!labeled V a}.
  !labeled U K, unlabeled U -o {one}.
\end{verbatim}\end{small}

Finally we have a rule to generate the symmetric closure
of the initial edge relation, again relying on saturation.

\begin{small}\begin{verbatim}
  !edge U V -o {!edge V U}.
\end{verbatim}\end{small}

This completes the implementation.  An interesting aspect of it is the
use of linearity to avoid explicit deletions.  These are present in
Ganzinger and McAllester's specification, but have no discernible
logical justification.

\section{Conclusion}
\label{sec:conclusion}

We have presented a programming language, LolliMon, based on intuitionistic
linear logic augmented with a monad.  Computation in LolliMon proceeds via proof
construction.  In the asynchronous fragment (outside the monad),
computation uses backward chaining and backtracking as in the Lolli
language~\cite{Hodas94ic}.  In the synchronous fragment (inside the
monad) computation uses forward chaining and committed choice, which
allows natural models of concurrency and saturation-based
algorithms.  The interaction between these strategies is rich,
yet the monadic structure of the underlying logic keeps it manageable.
We have presented several examples, and more are provided with the
implementation.

On the logical side, the first items of future work are to gain a better
understanding of additive disjunction ($\oplus$) and falsehood
($\zero$), and to extend the operational semantics to the fully
dependent type theory CLF~\cite{Watkins02tr}.

On the computational side, we plan to consider issues of fair scheduling
and the difficulties that arise from free variables during forward
chaining and, in particular, with respect to saturation.  We also plan to
give a more thorough analysis of unification and unification
constraints.

As far as the implementation is concerned, the most pressing need is to
devise techniques for more efficient forward chaining.

Finally, we are interested in developing analysis tools for LolliMon
programs, such as mode or termination checkers.  This should be
particularly interesting for forward chaining, where it may be possible
to use model checking for state-space exploration to establish temporal
properties of LolliMon program executions, or to support automatic
complexity analysis in the style advocated by Ganzinger and
McAllester~\cite{Ganzinger01ijcar}.

\ignore{
\bibliographystyle{abbrv}
\bibliography{ppdp05}
}

\begin{thebibliography}{50}
\vspace*{0.5mm}
\scriptsize

\bibitem{Andreoli92}
J.-M. Andreoli.
\newblock Logic programming with focusing proofs in linear logic.
\newblock {\em Journal of Logic and Computation}, 2(3):297--347, 1992.

\bibitem{Andreoli90oopsla}
J.-M. Andreoli and R.~Pareschi.
\newblock {LO} and behold! {C}oncurrent structured processes.
\newblock In {\em {OOPSLA/ECOOP} '90: {P}roceedings on object-oriented
  programming systems, languages, and applications}, pages 44--56, Ottawa,
  Canada, 1990. ACM Press.

\bibitem{Bekkers95ilps}
Y.~Bekkers and P.~Tarau.
\newblock Monadic constructs for logic programming.
\newblock In J.Lloyd, editor, {\em Proceedings of the International Symposium
  on Logic Programming (ILPS'95)}, pages 51--65, Portland, Oregon, Dec. 1995.
  MIT Press.

\bibitem{Bozzano04tplp}
M.~Bozzano, G.~Delzanno, and M.~Martelli.
\newblock Model checking linear logic specifications.
\newblock {\em Theory and Practice of Logic Programming}, 4(5--6):573--619,
  2004.

\bibitem{Cervesato02tr}
I.~Cervesato, F.~Pfenning, D.~Walker, and K.~Watkins.
\newblock A concurrent logical framework {II}: Examples and applications.
\newblock Technical Report CMU-CS-02-102, Department of Computer Science,
  Carnegie Mellon University, 2002.
\newblock Revised May 2003.

\bibitem{Fairtlough97ic}
M.~Fairtlough and M.~Mendler.
\newblock Propositional lax logic.
\newblock {\em Information and Computation}, 137(1):1--33, Aug. 1997.

\bibitem{Fruehwirth98jlp}
T.~Fr{\"u}hwirth.
\newblock Theory and practice of constraint handling rules.
\newblock {\em Journal of Logic Programming}, 37(1--3):95--138, Oct. 1998.
\newblock Special Issue on Constraint Logic Programming.

\bibitem{Ganzinger01ijcar}
H.~Ganzinger and D.~A. McAllester.
\newblock A new meta-complexity theorem for bottom-up logic programs.
\newblock In T.~R.Gor{\'e}, A.Leitsch, editor, {\em Proceedings of the First
  International Joint Conference on ArAutomated Reasoning (IJCAR'01)}, pages
  514--528, Siena, Italy, June 2001. Springer-Verlag LNCS 2083.

\bibitem{Ganzinger02iclp}
H.~Ganzinger and D.~A. McAllester.
\newblock Logical algorithms.
\newblock In P.Stuckey, editor, {\em Proceedings of the 18th International
  Conference on Logic Programming}, pages 209--223, Copenhagen, Denmark, July
  2002. Springer-Verlag LNCS 2401.

\bibitem{Girard87tcs}
J.-Y. Girard.
\newblock Linear logic.
\newblock {\em Theoretical Computer Science}, 50:1--102, 1987.

\bibitem{Halstead85}
R.~H. Halstead.
\newblock {Multilisp}: A language for parallel symbolic computation.
\newblock {\em ACM Transactions on Programming Languages and Systems},
  7(4):501--539, Oct. 1985.

\bibitem{Harland00pstt}
J.~Harland, D.~Pym, and M.~Winikoff.
\newblock Forward and backward chaining in linear logic.
\newblock In D.~Galmiche, editor, {\em Proceedings of the Workshop on
  Type-Theoretic Languages: Proof Search and Semantics}, volume~37 of {\em
  Electronic Notes in Theoretical Computer Science}, Pittsburgh, Pennsylvania,
  June 2000. Elsevier Science.

\bibitem{Hodas94ic}
J.~Hodas and D.~Miller.
\newblock Logic programming in a fragment of intuitionistic linear logic.
\newblock {\em Information and Computation}, 110(2):327--365, 1994.

\bibitem{Kobayashi94tr}
N.~Kobayashi and A.~Yonezawa.
\newblock Typed higher-order concurrent linear logic programming.
\newblock Technical Report 94-12, Department of Information Science, University
  of Tokyo, July 1994.

\bibitem{LopPol05lpar}
P.~L\'{o}pez and J.~Polakow.
\newblock Implementing efficient resource management for linear logic
  programming.
\newblock In F.~Baader and A.~Voronkov, editors, {\em Eleventh International
  Conference on Logic for Programming Artificial Intelligence and Reasoning
  (LPAR'05)}, pages 528--543, Montevideo, Uruguay, Mar. 2005. Springer-Verlag
  LNAI 3452.

\bibitem{Mcallester02jacm}
D.~A. McAllester.
\newblock On the complexity analysis of static analyses.
\newblock {\em Journal of the ACM}, 49(4):512--537, 2002.

\bibitem{McGrail97}
R.~McGrail.
\newblock {\em Monads and Control in Logic Programming}.
\newblock PhD thesis, Wesleyan University, 1997.

\bibitem{Miller91jlc}
D.~Miller.
\newblock A logic programming language with lambda-abstraction, function
  variables, and simple unification.
\newblock {\em Journal of Logic and Computation}, 1(4):497--536, 1991.

\bibitem{Miller96tcs}
D.~Miller.
\newblock Forum: {A} multiple-conclusion specification logic.
\newblock {\em Theoretical Computer Science}, 165(1):201--232, 1996.

\bibitem{Miller86iclp}
D.~Miller and G.~Nadathur.
\newblock Higher-order logic programming.
\newblock In E.~Shapiro, editor, {\em Proceedings of the Third International
  Logic Programming Conference}, pages 448--462, London, June 1986.

\bibitem{Miller91apal}
D.~Miller, G.~Nadathur, F.~Pfenning, and A.~Scedrov.
\newblock Uniform proofs as a foundation for logic programming.
\newblock {\em Annals of Pure and Applied Logic}, 51:125--157, 1991.

\bibitem{Milner99}
R.~Milner.
\newblock {\em Communicating and Mobile Systems: the {$\pi$}-Calculus}.
\newblock Cambridge University Press, 1999.

\bibitem{Moggi91}
E.~Moggi.
\newblock Notions of computation and monads.
\newblock {\em Information and Computation}, 93(1):55--92, 1991.

\bibitem{Pfenning04aplas}
F.~Pfenning.
\newblock Substructural operational semantics and linear destination-passing
  style.
\newblock In W.-N. Chin, editor, {\em Proceedings of the 2nd Asian Symposium on
  Programming Languages and Systems (APLAS'04)}, page 196, Taipei, Taiwan, Nov.
  2004. Springer-Verlag LNCS 3302.

\bibitem{Pfenning01mscs}
F.~Pfenning and R.~Davies.
\newblock A judgmental reconstruction of modal logic.
\newblock {\em Mathematical Structures in Computer Science}, 11:511--540, 2001.

\bibitem{PymHarland94}
D.~Pym and J.~Harland.
\newblock A uniform proof-theoretic investigation of linear logic programming.
\newblock {\em Journal of Logic and Computation}, 4(2):175--207, April 1994.

\bibitem{RSV01}
I.~Ramakrishnan, R.~Sekar, and A.~Voronkov.
\newblock Term indexing.
\newblock In {\em Handbook of Automated Reasoning, vol. 2}, pages 1853--1965.
  Elsevier Science and MIT Press, 2001.

\bibitem{Shapiro89}
E.~Y. Shapiro.
\newblock The family of concurrent logic programming languages.
\newblock {\em ACM Computing Surveys}, 21(3):413--510, Sept. 1989.

\bibitem{Wadler92popl}
P.~Wadler.
\newblock The essence of functional programming.
\newblock In {\em Conference Record of the 19th Symposium on Principles of
  Programming Languages}, pages 1--14, Albuquerque, Jan. 1992. ACM Press.

\bibitem{Watkins02tr}
K.~Watkins, I.~Cervesato, F.~Pfenning, and D.~Walker.
\newblock A concurrent logical framework {I}: Judgments and properties.
\newblock Technical Report CMU-CS-02-101, Department of Computer Science,
  Carnegie Mellon University, 2002.
\newblock Revised May 2003.

\bibitem{Watkins04types}
K.~Watkins, I.~Cervesato, F.~Pfenning, and D.~Walker.
\newblock A concurrent logical framework: The propositional fragment.
\newblock In S.~Berardi, M.~Coppo, and F.~Damiani, editors, {\em Types for
  Proofs and Programs}, pages 355--377. Springer-Verlag LNCS 3085, 2004.
\newblock Revised selected papers from the {\em Third International Workshop on
  Types for Proofs and Programs}, Torino, Italy, April 2003.

\end{thebibliography}

\appendix
\section*{Logical Rules Summary}
\label{apx:logical-rules-summary}
\onecolumn
\noindent Syntax:
$$
\begin{array}{lll}
A & ::= &  P 
      \mid \top 
      \mid A_1 \with A_2 
      \mid \\&& A_1 \limp A_2 
      \mid A_1 \iimp A_2 
      \mid \forall x{:}\tau.A  
      \mid \{ S \} 
\\[1ex]
S & ::= &  A
      \mid \bang A
      \mid \one 
      \mid S_1 \tensor S_2 
      \mid \exists x{:}\tau.S 
\end{array}
\qquad
\begin{array}{lll}
  \Gamma & ::= & \cdot \mid \Gamma, A \\
  \Delta & ::= & \cdot \mid \Delta, A \\%[2ex]
  \Psi   & ::= & \cdot \mid S, \Psi \\
\end{array}
$$

\noindent Sequent forms:
$$
\begin{array}{ll}
  \CLFRIseq{\Gamma}{\Delta}{A} & \hbox{Right inversion} \\
  \CLFLFseq{\Gamma}{\Delta}{A}{P} & \hbox{Left focusing} \\
\end{array}
\qquad
\begin{array}{ll}
  \CLFLseq{\Gamma}{\Delta}{S} & \hbox{Forward chaining} \\
  \CLFLLFseq{\Gamma}{\Delta}{A}{\{S\}} & \hbox{Monadic left focusing} \\
  \CLFLIseq{\Gamma}{\Delta}{\Psi}{S} & \hbox{Left inversion} \\
  \CLFRFseq{\Gamma}{\Delta}{S} & \hbox{Right focusing} \\
\end{array}
$$
\end{document}

Rules for backward chaining:
$$
\nsrule{\CLFLFseq{\Gamma,A}{\Delta}{A}{P}}
       {\CLFRIseq{\Gamma,A}{\Delta}{P}}
       {\GRuleName{\CLFsystem}{\hbox{uhyp}}}
\qquad
\nsrule{\CLFLFseq{\Gamma}{\Delta}{A}{P}}
       {\CLFRIseq{\Gamma}{\Delta,A}{P}}
       {\GRuleName{\CLFsystem}{\hbox{lhyp}}}
\qquad
\nsrule{}
       {\CLFLFseq{\Gamma}{\cdot}{P}{P}}
       {\GRuleName{\CLFsystem}{\hbox{atm}}}
\qquad
\nsrule{\CLFLseq{\Gamma}{\Delta}{S}}
       {\CLFRIseq{\Gamma}{\Delta}{\{S\}}}
       {\GRuleName{\CLFsystem}{\{\}_R}}
$$
$$
\nsrule{\CLFRIseq{\Gamma,A}{\Delta}{B}}
       {\CLFRIseq{\Gamma}{\Delta}{A \iimp B}}
       {\GRuleName{\CLFsystem}{\iimp_R}}
\qquad
\nsrule{\CLFLFseq{\Gamma}{\Delta}{B}{P}
        \andalso
        \CLFRIseq{\Gamma}{\cdot}{A}}
       {\CLFLFseq{\Gamma}{\Delta}{A\iimp B}{P}}
       {\GRuleName{\CLFsystem}{\iimp_L}}
$$
$$
\nsrule{\CLFRIseq{\Gamma}{\Delta,A}{B}}
       {\CLFRIseq{\Gamma}{\Delta}{A\limp B}}
       {\GRuleName{\CLFsystem}{\limp_R}}
\qquad
\nsrule{\CLFLFseq{\Gamma}{\Delta_1}{B}{P}
        \andalso
        \CLFRIseq{\Gamma}{\Delta_2}{A}}
       {\CLFLFseq{\Gamma}{\Delta_1,\Delta_2}{A\limp B}{P}}
       {\GRuleName{\CLFsystem}{\limp_L}}
$$
$$
\nsrule{}
       {\CLFRIseq{\Gamma}{\Delta}{\top}}
       {\GRuleName{\CLFsystem}{\top_R}}
\qquad
\hbox{(no $\top_L$ rule)}
$$
$$
\nsrule{\CLFRIseq{\Gamma}{\Delta}{A}
        \andalso
        \CLFRIseq{\Gamma}{\Delta}{B}}
       {\CLFRIseq{\Gamma}{\Delta}{A\with B}}
       {\GRuleName{\CLFsystem}{\with_R}}
\qquad
\nsrule{\CLFLFseq{\Gamma}{\Delta}{A}{P}}
       {\CLFLFseq{\Gamma}{\Delta}{A\with B}{P}}
       {\GRuleName{\CLFsystem}{\with_{L1}}}
\qquad
\nsrule{\CLFLFseq{\Gamma}{\Delta}{B}{P}}
       {\CLFLFseq{\Gamma}{\Delta}{A\with B}{P}}
       {\GRuleName{\CLFsystem}{\with_{L2}}}
$$
$$
\nsrule{\CLFRIseq{\Gamma}{\Delta}{[a/x]A}}
       {\CLFRIseq{\Gamma}{\Delta}{\forall x{:}\tau.A}}
       {\GRuleName{\CLFsystem}{\forall_R}}
\qquad
\nsrule{\CLFLFseq{\Gamma}{\Delta}{[t/x]A}{P}}
       {\CLFLFseq{\Gamma}{\Delta}{\forall x{:}\tau.A}{P}}
       {\GRuleName{\CLFsystem}{\forall_L}}
$$

Rules for forward chaining:
$$
\nsrule{\CLFLLFseq{\Gamma,A}{\Delta}{A}{S}}
       {\CLFLseq{\Gamma,A}{\Delta}{S}}
       {\GRuleName{\CLFsystem}{\hbox{uhyp}'}}
\qquad
\nsrule{\CLFLLFseq{\Gamma}{\Delta}{A}{S}}
       {\CLFLseq{\Gamma}{\Delta,A}{S}}
       {\GRuleName{\CLFsystem}{\hbox{lhyp}'}}
\qquad
\nsrule{\CLFLIseq{\Gamma}{\Delta}{S'}{S}}
       {\CLFLLFseq{\Gamma}{\Delta}{\{S'\}}{S}}
       {\GRuleName{\CLFsystem}{\{\}_L}}
\qquad
\nsrule{\CLFLseq{\Gamma}{\Delta}{S}}
       {\CLFLIseq{\Gamma}{\Delta}{\cdot}{S}}
       {\GRuleName{\CLFsystem}{\rightarrow\rightarrow}}
\qquad
\nsrule{\CLFRFseq{\Gamma}{\Delta}{S}}
       {\CLFLseq{\Gamma}{\Delta}{S}}
       {\GRuleName{\CLFsystem}{>\!\! >\rightarrow}}
$$
\qquad (Modified left rules for asynchronous connectives)
$$
\nsrule{\CLFLIseq{\Gamma}{\Delta,A}{\Psi}{S}}
       {\CLFLIseq{\Gamma}{\Delta}{A,\Psi}{S}}
       {\GRuleName{\CLFsystem}{\hbox{async}}}
\qquad
\nsrule{\CLFLFseq{\Gamma}{\Delta}{B}{P}
        \andalso
        \CLFRIseq{\Gamma}{\cdot}{A}}
       {\CLFLFseq{\Gamma}{\Delta}{A\iimp B}{P}}
       {\GRuleName{\CLFsystem}{\iimp_L}'}
\qquad
\nsrule{\CLFLFseq{\Gamma}{\Delta_1}{B}{P}
        \andalso
        \CLFRIseq{\Gamma}{\Delta_2}{A}}
       {\CLFLFseq{\Gamma}{\Delta_1,\Delta_2}{A\limp B}{P}}
       {\GRuleName{\CLFsystem}{\limp_L}'}
$$
$$
\nsrule{\CLFLFseq{\Gamma}{\Delta}{A}{P}}
       {\CLFLFseq{\Gamma}{\Delta}{A\with B}{P}}
       {\GRuleName{\CLFsystem}{\with_{L1}'}}
\qquad
\nsrule{\CLFLFseq{\Gamma}{\Delta}{B}{P}}
       {\CLFLFseq{\Gamma}{\Delta}{A\with B}{P}}
       {\GRuleName{\CLFsystem}{\with_{L2}'}}
\qquad
\nsrule{\CLFLFseq{\Gamma}{\Delta}{[t/x]A}{P}}
       {\CLFLFseq{\Gamma}{\Delta}{\forall x{:}\tau.A}{P}}
       {\GRuleName{\CLFsystem}{\forall_L}'}
\qquad
\hbox{(no $\top_L'$ rule)}
$$
$$
\nsrule{\CLFLIseq{\Gamma}{\Delta}{\Psi}{S}}
       {\CLFLIseq{\Gamma}{\Delta}{\one,\Psi}{S}}
       {\GRuleName{\CLFsystem}{\one_L}}
\qquad
\nsrule{\CLFLIseq{\Gamma}{\Delta}{S_1,S_2,\Psi}{S}}
       {\CLFLIseq{\Gamma}{\Delta}{S_1\tensor S_2,\Psi}{S}}
       {\GRuleName{\CLFsystem}{\tensor_L}}
\qquad
\nsrule{\CLFLIseq{\Gamma}{\Delta}{[a/x]S',\Psi}{S}}
       {\CLFLIseq{\Gamma}{\Delta}{\exists x{:}\tau.S',\Psi}{S}}
       {\GRuleName{\CLFsystem}{\exists_L}}
\qquad
\nsrule{\CLFLIseq{\Gamma,A}{\Delta}{\Psi}{S}}
       {\CLFLIseq{\Gamma}{\Delta}{\bang A,\Psi}{S}}
       {\GRuleName{\CLFsystem}{\bang_L}}
$$

Right rules for synchronous formulas:

$$
\nsrule{}
       {\CLFRFseq{\Gamma}{\cdot}{\one}}
       {\GRuleName{\CLFsystem}{\one_R}}
\qquad
\nsrule{\CLFRFseq{\Gamma}{\Delta_1}{S_1}
        \andalso
        \CLFRFseq{\Gamma}{\Delta_2}{S_2}}
       {\CLFRFseq{\Gamma}{\Delta_1,\Delta_2}{S_1\tensor S_2}}
       {\GRuleName{\CLFsystem}{\tensor_R}}
\qquad
\nsrule{\CLFRFseq{\Gamma}{\Delta}{[t/x]S}}
       {\CLFRFseq{\Gamma}{\Delta}{\exists x{:}\tau.S}}
       {\GRuleName{\CLFsystem}{\exists_R}}
\qquad
\nsrule{\CLFRIseq{\Gamma}{\cdot}{A}}
       {\CLFRFseq{\Gamma}{\cdot}{\bang A}}
       {\GRuleName{\CLFsystem}{\bang_R}}
\qquad
\nsrule{\CLFRIseq{\Gamma}{\Delta}{A}}
       {\CLFRFseq{\Gamma}{\Delta}{A}}
       {\GRuleName{\CLFsystem}{\Rightarrow>\!\!>}}
$$

\end{document}
